
\documentclass[letterpaper, 11pt]{extarticle}
% \usepackage{fontspec}

% ==================================================

% document parameters
% \usepackage[spanish, mexico, es-lcroman]{babel}
\usepackage[english]{babel}
\usepackage[margin = 1in]{geometry}

% ==================================================

% Packages for math
\usepackage{mathrsfs}
\usepackage{amsfonts}
\usepackage{amsmath}
\usepackage{amsthm}
\usepackage{amssymb}
\usepackage{physics}
\usepackage{dsfont}
\usepackage{esint}

% ==================================================

% Packages for writing
\usepackage{enumerate}
\usepackage[shortlabels]{enumitem}
\usepackage{framed}
\usepackage{csquotes}

% ==================================================

% Miscellaneous packages
\usepackage{float}
\usepackage{tabularx}
\usepackage{xcolor}
\usepackage{multicol}
\usepackage{subcaption}
\usepackage{caption}
\captionsetup{format = hang, margin = 10pt, font = small, labelfont = bf}

% Citation
\usepackage[round, authoryear]{natbib}

% Hyperlinks setup
\usepackage{hyperref}
\definecolor{links}{rgb}{0.36,0.54,0.66}
\hypersetup{
   colorlinks = true,
    linkcolor = black,
     urlcolor = blue,
    citecolor = blue,
    filecolor = blue,
    pdfauthor = {Author},
     pdftitle = {Title},
   pdfsubject = {subject},
  pdfkeywords = {one, two},
  pdfproducer = {LaTeX},
   pdfcreator = {pdfLaTeX},
   }
\usepackage{titlesec}
\usepackage[many]{tcolorbox}

% Adjust spacing after the chapter title
% \titlespacing*{<command>}{<left>}{<before-sep>}{<after-sep>}
\titlespacing*{\chapter}{0cm}{-2.0cm}{0.50cm}
\titlespacing*{\section}{0cm}{0.50cm}{0.25cm}

% Indent 
\setlength{\parindent}{0pt}
\setlength{\parskip}{1ex}

% --- Theorems, lemma, corollary, postulate, definition ---
% \numberwithin{equation}{section}


\newtcbtheorem[]{problem}{Problem}%
    {enhanced,
    colback = black!5, %white,
    colbacktitle = black!5,
    coltitle = black,
    boxrule = 0pt,
    frame hidden,
    borderline west = {0.5mm}{0.0mm}{black},
    fonttitle = \bfseries\sffamily,
    breakable,
    before skip = 3ex,
    after skip = 3ex
}{problem}

\tcbuselibrary{skins, breakable}
%%% Commands defined by me
%   Euler constant
\newcommand{\eu}{\mathrm{e}}

%   Imaginary unit
\newcommand{\im}{\mathrm{i}}

%   Degrees
\newcommand{\grado}{\,^{\circ}}

%%% Linear Algebra

% Transpose
\newcommand{\transpose}[1]{{#1}^{\mathsf{T}}}

%%% Calculus
%   Integral from - infinity to infinity 
\newcommand{\Int}{\int\limits_{-\infty}^{\infty}}

%   Indefinite integral
\newcommand{\rint}[2]{\int{#1}\dd{#2}}

%   Definite integral
\newcommand{\Rint}[4]{\int\limits_{#1}^{#2}{#3}\dd{#4}}


% Serif bold text
\newcommand{\tsb}[1]{\textsf{\textbf{#1}}}

% Separation line
\newcommand{\linea}{\textcolor{gray!60}{\rule{\linewidth}{0.2pt}}}

% Bigger version of a \cdot to denote dot product
\makeatletter
\newcommand*\bigcdot{\mathpalette\bigcdot@{.5}}
\newcommand*\bigcdot@[2]{\mathbin{\vcenter{\hbox{\scalebox{#2}{$\m@th#1\bullet$}}}}}
\makeatother

% My Hamiltonian prefered notation
\newcommand{\Ham}{\hat{\mathcal{H}}}

%% Pre-existing commands redefined by me
% Trace of a matrix 
\renewcommand{\Tr}{\mathrm{Tr}}



\begin{document}
	\begin{Large}
		\textsf{\textbf{Find norming constant of lognormal distribution}}\\
		Section 21 - Home Work 2
	\end{Large}
	
	\vspace{1ex}
	
	\textsf{\textbf{Student:}} \text{Mehrab Atighi}, \href{mailto:mehrab.atighi@gmail.com}{\texttt{mehrab.atighi@gmail.com}}\\
	\textsf{\textbf{Lecturer:}} \text{Mohammad Zokaei}, \href{mailto:Zokaei@sbu.ac.ir}{\texttt{Zokaei@sbu.ac.ir}}
	
	
	\vspace{2ex}
	
	\begin{problem}{}{problem-label}
		\textbf{According to the Example 3.3.31 find norming constant of lognormal distribution.}
		\cite{Embrechts.etal1997}
	\end{problem}
	
	\begin{solve}{}{solve-label}
	\subsection*{Definition of the Lognormal Distribution}
Let $X$ be a standard normal random variable, and define the lognormal random variable as:
\[
\tilde{X} = g(X) = e^{\mu + \sigma X}, \quad \mu \in \mathbb{R}, \ \sigma > 0.
\]
The goal is to study the asymptotic behavior of the maximum of the lognormal distribution, $\tilde{X}$, and show that $\tilde{X} \in \text{MDA}(\Lambda)$, the maximum domain of attraction of the Gumbel distribution.

\subsection*{The Standard Normal Maximum}
Since $X \in \text{MDA}(\Lambda)$, the maximum $M_n$ of $n$ standard normal random variables satisfies:
\[
\lim_{n \to \infty} P\left( \frac{M_n - d_n}{c_n} \leq x \right) = \Lambda(x), \quad x \in \mathbb{R},
\]
where $c_n$ and $d_n$ are norming constants for the standard normal distribution, given as:
\[
c_n \sim \frac{1}{(2 \ln n)^{1/2}}, \quad d_n \sim (2 \ln n)^{1/2} - \frac{\ln \ln n + \ln(4\pi)}{2(2 \ln n)^{1/2}}.
\]
Now, expanding the form of $d_n$ for large $n$, we have:
\[
d_n \sim (2 \ln n)^{1/2} - \frac{\ln \ln n + \ln(4\pi)}{2(2 \ln n)^{1/2}}.
\]
We can write the expansion of $d_n$ as:
\[
d_n = (2 \ln n)^{1/2} \left( 1 - \frac{\ln \ln n + \ln(4\pi)}{2 (2 \ln n)^{1/2}} \right).
\]
As $n \to \infty$, we see that $d_n \sim (2 \ln n)^{1/2}$, and the correction term is of smaller order.

\subsection*{The Lognormal Maximum}
Applying the transformation $g(x) = e^{\mu + \sigma x}$ to the maximum, the transformed maximum is:
\[
\tilde{M}_n = g(M_n) = e^{\mu + \sigma M_n}.
\]
To standardize $\tilde{M}_n$, we analyze the probability:
\[
P\left( \tilde{M}_n \leq e^{\mu + \sigma (c_n x + d_n)} \right).
\]
Taking the logarithm of both sides gives:
\[
P\left( M_n \leq c_n x + d_n \right),
\]
which converges to the Gumbel distribution:
\[
P\left( \frac{M_n - d_n}{c_n} \leq x \right) \to \Lambda(x), \quad \text{as } n \to \infty.
\]

\subsection*{Derivation of Norming Constants}
Returning to $\tilde{M}_n$, we approximate it as:
\[
\tilde{M}_n = e^{\mu + \sigma M_n} = e^{\mu + \sigma d_n} e^{\sigma c_n x + o(\sigma c_n)}.
\]
Using the first-order approximation for the exponential term:
\[
e^{\sigma c_n x + o(\sigma c_n)} \approx 1 + \sigma c_n x + o(\sigma c_n),
\]
we find:
\[
\tilde{M}_n \approx e^{\mu + \sigma d_n} \left( 1 + \sigma c_n x + o(\sigma c_n) \right).
\]
Thus, we approximate the norming constants:
\[
\tilde{c}_n = \sigma c_n e^{\mu + \sigma d_n}, \quad \tilde{d}_n = e^{\mu + \sigma d_n}.
\]

\subsection*{Final Results}
The lognormal maximum $\tilde{M}_n$ satisfies:
\[
\lim_{n \to \infty} P\left( \frac{\tilde{M}_n - \tilde{d}_n}{\tilde{c}_n} \leq x \right) = \Lambda(x), \quad x \in \mathbb{R}.
\]
Thus, $\tilde{X} \in \text{MDA}(\Lambda)$, with norming constants:
\[
\tilde{c}_n = \sigma c_n e^{\mu + \sigma d_n}, \quad \tilde{d}_n = e^{\mu + \sigma d_n}.
\]
		
		\cite{Embrechts.etal1997}
	\end{solve}
	% =================================================
	
	% \newpage
	
	% \vfill
	
	\bibliographystyle{apalike}
	\bibliography{references}
\end{document}