	
\documentclass[letterpaper, 11pt]{extarticle}
% \usepackage{fontspec}

% ==================================================

% document parameters
% \usepackage[spanish, mexico, es-lcroman]{babel}
\usepackage[english]{babel}
\usepackage[margin = 1in]{geometry}

% ==================================================

% Packages for math
\usepackage{mathrsfs}
\usepackage{amsfonts}
\usepackage{amsmath}
\usepackage{amsthm}
\usepackage{amssymb}
\usepackage{physics}
\usepackage{dsfont}
\usepackage{esint}

% ==================================================

% Packages for writing
\usepackage{enumerate}
\usepackage[shortlabels]{enumitem}
\usepackage{framed}
\usepackage{csquotes}

% ==================================================

% Miscellaneous packages
\usepackage{float}
\usepackage{tabularx}
\usepackage{xcolor}
\usepackage{multicol}
\usepackage{subcaption}
\usepackage{caption}
\captionsetup{format = hang, margin = 10pt, font = small, labelfont = bf}

% Citation
\usepackage[round, authoryear]{natbib}

% Hyperlinks setup
\usepackage{hyperref}
\definecolor{links}{rgb}{0.36,0.54,0.66}
\hypersetup{
   colorlinks = true,
    linkcolor = black,
     urlcolor = blue,
    citecolor = blue,
    filecolor = blue,
    pdfauthor = {Author},
     pdftitle = {Title},
   pdfsubject = {subject},
  pdfkeywords = {one, two},
  pdfproducer = {LaTeX},
   pdfcreator = {pdfLaTeX},
   }
\usepackage{titlesec}
\usepackage[many]{tcolorbox}

% Adjust spacing after the chapter title
% \titlespacing*{<command>}{<left>}{<before-sep>}{<after-sep>}
\titlespacing*{\chapter}{0cm}{-2.0cm}{0.50cm}
\titlespacing*{\section}{0cm}{0.50cm}{0.25cm}

% Indent 
\setlength{\parindent}{0pt}
\setlength{\parskip}{1ex}

% --- Theorems, lemma, corollary, postulate, definition ---
% \numberwithin{equation}{section}


\newtcbtheorem[]{problem}{Problem}%
    {enhanced,
    colback = black!5, %white,
    colbacktitle = black!5,
    coltitle = black,
    boxrule = 0pt,
    frame hidden,
    borderline west = {0.5mm}{0.0mm}{black},
    fonttitle = \bfseries\sffamily,
    breakable,
    before skip = 3ex,
    after skip = 3ex
}{problem}

\tcbuselibrary{skins, breakable}
%%% Commands defined by me
%   Euler constant
\newcommand{\eu}{\mathrm{e}}

%   Imaginary unit
\newcommand{\im}{\mathrm{i}}

%   Degrees
\newcommand{\grado}{\,^{\circ}}

%%% Linear Algebra

% Transpose
\newcommand{\transpose}[1]{{#1}^{\mathsf{T}}}

%%% Calculus
%   Integral from - infinity to infinity 
\newcommand{\Int}{\int\limits_{-\infty}^{\infty}}

%   Indefinite integral
\newcommand{\rint}[2]{\int{#1}\dd{#2}}

%   Definite integral
\newcommand{\Rint}[4]{\int\limits_{#1}^{#2}{#3}\dd{#4}}


% Serif bold text
\newcommand{\tsb}[1]{\textsf{\textbf{#1}}}

% Separation line
\newcommand{\linea}{\textcolor{gray!60}{\rule{\linewidth}{0.2pt}}}

% Bigger version of a \cdot to denote dot product
\makeatletter
\newcommand*\bigcdot{\mathpalette\bigcdot@{.5}}
\newcommand*\bigcdot@[2]{\mathbin{\vcenter{\hbox{\scalebox{#2}{$\m@th#1\bullet$}}}}}
\makeatother

% My Hamiltonian prefered notation
\newcommand{\Ham}{\hat{\mathcal{H}}}

%% Pre-existing commands redefined by me
% Trace of a matrix 
\renewcommand{\Tr}{\mathrm{Tr}}



\begin{document}
		\begin{Large}
		\textsf{\textbf{Expected Value of Risk Process}}\\
		Section 3 - Home Work 1
	\end{Large}
	
	\vspace{1ex}
	
	\textsf{\textbf{Student:}} \text{Mehrab Atighi}, \href{mailto:mehrab.atighi@gmail.com}{\texttt{mehrab.atighi@gmail.com}}\\
	\textsf{\textbf{Lecturer:}} \text{Mohammad Zokaei}, \href{mailto:Zokaei@sbu.ac.ir}{\texttt{Zokaei@sbu.ac.ir}}
	
	
	\vspace{2ex}
	
	\begin{problem}{}{problem-label}
		Proof of the Expected Value of the Risk Process with Small \( o \) Term. \cite{Embrechts.etal1997}:

		\begin{enumerate}[(a)]
			\item Prove the following
			\begin{enumerate}[label = (\roman*)]
				\item $E[U(t)] = u + (c - \lambda \mu)t (1 + o(1))$
			\end{enumerate}
		\end{enumerate}
	\end{problem}
	
	\begin{solve}{}{solve-label}
		To prove the expected value of the risk process \( U(t) \) in a renewal model, we start with the definition:
		$$U(t) = u + ct - \sum_{i=1}^{N(t)} X_i$$
		where:
		\begin{enumerate}
		\item \( u \) is the initial reserve.
		\item \( c \) is the premium rate.
		\item \( N(t) \) is the number of claims up to time \( t \), which follows a renewal process.
		\item  \( X_i \) are the individual claim amounts, assumed to be i.i.d. random variables with mean \( \mu \). 
	\end{enumerate}
	Steps to Prove the Expected Value
	\begin{enumerate}
		
		\item \textbf{Define the Renewal Process:}
		\subitem Let \( \{X_i\} \) be the interarrival times between claims, which are i.i.d. with distribution function \( F_T \) and mean \( \lambda^{-1} \).
		\subitem The counting process \( N(t) \) represents the number of claims by time \( t \).
		
		\item \textbf{Risk Process:}
		\subitem The risk process \( U(t) \) is given by:
		$$U(t) = u + ct - \sum_{i=1}^{N(t)} X_i$$
		
		\item \textbf{Expected Value of the Risk Process:}
		\subitem To find the expected value \( E[U(t)] \), we use the linearity of expectation:
		$$E[U(t)] = E[u + ct - \sum_{i=1}^{N(t)} X_i]$$
		$$E[U(t)] = u + ct - E[\sum_{i=1}^{N(t)} X_i]$$
		
		\item \textbf{Expected Value of the Sum of Claims:}
		\subitem The expected value of the sum of claims up to time \( t \) is:
		$$E[\sum_{i=1}^{N(t)} X_i] = E[N(t)] \cdot E[X]$$
		\subitem Using the renewal reward theorem, we know that for large \( t \):
		$$E[N(t)] \approx \frac{t}{\lambda}$$
		\subitem Therefore:
		$$E[\sum_{i=1}^{N(t)} X_i] \approx \frac{t}{\lambda} \cdot \mu$$
		
		\item \textbf{Incorporating the Small \( o \) Term}
		\subitem The renewal reward theorem also implies that there is a small error term \( o(1) \) that goes to zero as \( t \) grows large. Thus, we can write:
		$$E[N(t)] = \frac{t}{\lambda} (1 + o(1))$$
		\subitem Therefore:
		$$E[\sum_{i=1}^{N(t)} X_i] = \frac{t}{\lambda} \cdot \mu (1 + o(1))$$
		
		\item \textbf{Combining Results:}
		\subitem Substituting back into the expression for \( E[U(t)] \):
		$$E[U(t)] = u + ct - \frac{t}{\lambda} \cdot \mu (1 + o(1))$$
		\subitem Simplifying, we get:
		$$E[U(t)] = u + ct - \frac{\mu}{\lambda} t (1 + o(1))$$
		\subitem Since $(\lambda = \frac{1}{\mu})$, we have:
		$$E[U(t)] = u + (c - \lambda \mu)t (1 + o(1))$$
	\end{enumerate}
		\textbf{Conclusion:}\\
		The expected value of the risk process \( U(t) \) in a renewal model, as \( t \) approaches infinity, is given by:
		$$E[U(t)] = u + (c - \lambda \mu)t (1 + o(1))$$
		This incorporates the small \( o \) term, which represents a small error that diminishes as \( t \) becomes large. \cite{r1,r2,r3,r4,r5}
	\end{solve}
	% =================================================
	
	% \newpage
	
	% \vfill
	
	\bibliographystyle{apalike}
	\bibliography{references}
\end{document}