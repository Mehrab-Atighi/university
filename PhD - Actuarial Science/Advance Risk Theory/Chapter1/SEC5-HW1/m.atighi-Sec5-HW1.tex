	
\documentclass[letterpaper, 11pt]{extarticle}
% \usepackage{fontspec}

% ==================================================

% document parameters
% \usepackage[spanish, mexico, es-lcroman]{babel}
\usepackage[english]{babel}
\usepackage[margin = 1in]{geometry}

% ==================================================

% Packages for math
\usepackage{mathrsfs}
\usepackage{amsfonts}
\usepackage{amsmath}
\usepackage{amsthm}
\usepackage{amssymb}
\usepackage{physics}
\usepackage{dsfont}
\usepackage{esint}

% ==================================================

% Packages for writing
\usepackage{enumerate}
\usepackage[shortlabels]{enumitem}
\usepackage{framed}
\usepackage{csquotes}

% ==================================================

% Miscellaneous packages
\usepackage{float}
\usepackage{tabularx}
\usepackage{xcolor}
\usepackage{multicol}
\usepackage{subcaption}
\usepackage{caption}
\captionsetup{format = hang, margin = 10pt, font = small, labelfont = bf}

% Citation
\usepackage[round, authoryear]{natbib}

% Hyperlinks setup
\usepackage{hyperref}
\definecolor{links}{rgb}{0.36,0.54,0.66}
\hypersetup{
   colorlinks = true,
    linkcolor = black,
     urlcolor = blue,
    citecolor = blue,
    filecolor = blue,
    pdfauthor = {Author},
     pdftitle = {Title},
   pdfsubject = {subject},
  pdfkeywords = {one, two},
  pdfproducer = {LaTeX},
   pdfcreator = {pdfLaTeX},
   }
\usepackage{titlesec}
\usepackage[many]{tcolorbox}

% Adjust spacing after the chapter title
% \titlespacing*{<command>}{<left>}{<before-sep>}{<after-sep>}
\titlespacing*{\chapter}{0cm}{-2.0cm}{0.50cm}
\titlespacing*{\section}{0cm}{0.50cm}{0.25cm}

% Indent 
\setlength{\parindent}{0pt}
\setlength{\parskip}{1ex}

% --- Theorems, lemma, corollary, postulate, definition ---
% \numberwithin{equation}{section}


\newtcbtheorem[]{problem}{Problem}%
    {enhanced,
    colback = black!5, %white,
    colbacktitle = black!5,
    coltitle = black,
    boxrule = 0pt,
    frame hidden,
    borderline west = {0.5mm}{0.0mm}{black},
    fonttitle = \bfseries\sffamily,
    breakable,
    before skip = 3ex,
    after skip = 3ex
}{problem}

\tcbuselibrary{skins, breakable}
%%% Commands defined by me
%   Euler constant
\newcommand{\eu}{\mathrm{e}}

%   Imaginary unit
\newcommand{\im}{\mathrm{i}}

%   Degrees
\newcommand{\grado}{\,^{\circ}}

%%% Linear Algebra

% Transpose
\newcommand{\transpose}[1]{{#1}^{\mathsf{T}}}

%%% Calculus
%   Integral from - infinity to infinity 
\newcommand{\Int}{\int\limits_{-\infty}^{\infty}}

%   Indefinite integral
\newcommand{\rint}[2]{\int{#1}\dd{#2}}

%   Definite integral
\newcommand{\Rint}[4]{\int\limits_{#1}^{#2}{#3}\dd{#4}}


% Serif bold text
\newcommand{\tsb}[1]{\textsf{\textbf{#1}}}

% Separation line
\newcommand{\linea}{\textcolor{gray!60}{\rule{\linewidth}{0.2pt}}}

% Bigger version of a \cdot to denote dot product
\makeatletter
\newcommand*\bigcdot{\mathpalette\bigcdot@{.5}}
\newcommand*\bigcdot@[2]{\mathbin{\vcenter{\hbox{\scalebox{#2}{$\m@th#1\bullet$}}}}}
\makeatother

% My Hamiltonian prefered notation
\newcommand{\Ham}{\hat{\mathcal{H}}}

%% Pre-existing commands redefined by me
% Trace of a matrix 
\renewcommand{\Tr}{\mathrm{Tr}}



\begin{document}
		\begin{Large}
		\textsf{\textbf{Proof of corollary 1.3.2}}\\
		Section 5 - Home Work 1
	\end{Large}
	
	\vspace{1ex}
	
	\textsf{\textbf{Student:}} \text{Mehrab Atighi}, \href{mailto:mehrab.atighi@gmail.com}{\texttt{mehrab.atighi@gmail.com}}\\
	\textsf{\textbf{Lecturer:}} \text{Mohammad Zokaei}, \href{mailto:Zokaei@sbu.ac.ir}{\texttt{Zokaei@sbu.ac.ir}}
	
	
	\vspace{2ex}
	
	\begin{problem}{}{problem-label}
		Proof of bellow relation \cite{Embrechts.etal1997}:

		\begin{enumerate}[(a)]
			\item Prove the following
			\begin{enumerate}[label = (\roman*)]
				\item If $F(x) = -x^{\alpha} L(x)$ for $\alpha > 0$  and $L \in R_{\rho}$ , then for all $n \geq 1$ ,
				
				$$\bar{F}^{n*}(x) \sim n \bar{F}(x), \quad x \rightarrow \infty$$
				
			\end{enumerate}
		\end{enumerate}
	\end{problem}
	
	\begin{solve}{}{solve-label}
		To solve this problem, I will take the help of inductive proof, which will be discussed further:
		Suppose now that $X_1 , \cdots , X_n$ are iid with df F as in the above corollary Denote the partial sum of $X_1 , \cdots , X_n$ by $S_n = X_1 + \cdot + X_n$ and their maximum by $M_n = max(X_1 , \cdots , X_n).$ Then for all $n \geq 2$,
		$$P(S_n>x) = \bar{F^{n*}}(x) ,$$
		$$P(M_n >x) = \bar{F}^n(x) = \bar{F}(x) \sum_{k=0}^{n-1} F^{k}(x) \sim n \bar{F}(x)$$
		so we start with set n = 2 and we have:
		$$P(M_2 > x) = 1 - P(M_2 \leq x) = 1 - P(X_1 <x) P(X_2 <x)$$
		$$1 - (F(x))^2 = (1-F(x))(1+F(x)) $$
		$$\bar{F}(x)(1+F(x)) = \bar{F}(x)\sum_{k=0}^{n-1}F^k(x)$$
		and now we should assume that n = k is usable for proofing n = k+1 so we have:
		$$P(M_{n+1} > x) = 1-P(M_{n+1} <x) = 1-P(M_1<x) P(M_n<x)$$
		$$= 1-F(x)(1-P(M_n>x)) = 1-F(x)(1-\bar{F}(x)\sum_{k=0}^{n-1}F^k(x)) $$
		$$= 1 - (F(x) + F(x)\bar{F}(x)\sum_{k=0}^{n-1}F^k(x)) = 1 - F(x) + F(x)\bar{F}(x)\sum_{k=0}^{n-1}F^k(x)$$
		$$= \bar{F}(x) + F(x)\bar{F}(x)\sum_{k=0}^{n-1}F^k(x) = \bar{F}(x)[1+F(x)\sum_{k=0}^{n-1}F^k(x)] = \bar{F}(x)[\sum_{k=0}^{n}F^k(x)]$$
		$$\rightarrowtail \lim_{x \to \infty}\bar{F}(x)\sum_{k=0}^{n}F^k(x) = \bar{F}(x) * n$$
		Done. so we can say the assumed where n = k is ok was write.
	\end{solve}
	% =================================================
	
	% \newpage
	
	% \vfill
	
	\bibliographystyle{apalike}
	\bibliography{references}
\end{document}