	
\input{preamble}
\usepackage{titlesec}
\usepackage[many]{tcolorbox}
\usepackage{graphics}
\usepackage{float}
\usepackage{multirow}
\usepackage{float}
\usepackage{caption}
\usepackage{amssymb}
\usepackage{amsmath}
\usepackage{lineno}
\usepackage{algorithm}
\usepackage{algorithmic}
\usepackage{multirow}
\usepackage{graphics}
\usepackage{bm}
\usepackage{float}
\usepackage{listings}
\usepackage{color}
\usepackage{graphicx}



% Adjust spacing after the chapter title
% \titlespacing*{<command>}{<left>}{<before-sep>}{<after-sep>}
\titlespacing*{\chapter}{0cm}{-2.0cm}{0.50cm}
\titlespacing*{\section}{0cm}{0.50cm}{0.25cm}

% Indent 
\setlength{\parindent}{0pt}
\setlength{\parskip}{1ex}

% --- Theorems, lemma, corollary, postulate, definition ---
% \numberwithin{equation}{section}


\newtcbtheorem[]{problem}{Problem}%
    {enhanced,
    colback = black!5, %white,
    colbacktitle = black!5,
    coltitle = black,
    boxrule = 0pt,
    frame hidden,
    borderline west = {0.5mm}{0.0mm}{black},
    fonttitle = \bfseries\sffamily,
    breakable,
    before skip = 3ex,
    after skip = 3ex
}{problem}

\newtcbtheorem[]{solve}{Solve}%
{enhanced,
	colback = black!5, %white,
	colbacktitle = black,
	coltitle=white,
	boxrule = 0pt,
	frame hidden,
	borderline west = {0.5mm}{0.0mm}{black},
	fonttitle = \bfseries\sffamily,
	breakable,
	before skip = 3ex,
	after skip = 3ex
}{solve}

\tcbuselibrary{skins, breakable}
\input{commands}


\begin{document}
		\begin{Large}
		\textsf{\textbf{Domain of attraction of a normal distribution}}\\
		Section 10 - Home Work 3
	\end{Large}
	
	\vspace{1ex}
	
	\textsf{\textbf{Student:}} \text{Mehrab Atighi}, \href{mailto:mehrab.atighi@gmail.com}{\texttt{mehrab.atighi@gmail.com}}\\
	\textsf{\textbf{Lecturer:}} \text{Mohammad Zokaei}, \href{mailto:Zokaei@sbu.ac.ir}{\texttt{Zokaei@sbu.ac.ir}}
	
	
	\vspace{2ex}
	
	\begin{problem}{}{problem-label}
		\textbf{Proof of Corollary 2.2.9 (Domain of attraction of a normal distribution)}
		
		We are given a random variable \( X \) and aim to show that \( X \) is in the domain of attraction of a normal distribution if and only if one of the following conditions holds:
		\begin{enumerate}
			\item[(a)] \( \mathbb{E}[X^2] < \infty \),
			\item[(b)] \( \mathbb{E}[X^2] = \infty \) and 
			\[
			G(x) = P(|X| > x) = o \left( x^{-2} \int_{|y| \leq x} y^2 \, dF(y) \right), \quad x \to \infty.
			\]
		\end{enumerate}
		\cite{Embrechts.etal1997}:
	\end{problem}
	
	\begin{solve}{}{solve-label}
\textbf{Proof:}
We aim to show that \( X \) is in the domain of attraction of the normal distribution if and only if Condition (a) or Condition (b) holds.

\subsection*{Background: Domain of Attraction of a Normal Distribution}

A random variable \( X \) is said to be in the \textit{domain of attraction of a normal distribution} if, for a sequence of i.i.d.\ random variables \( X_1, X_2, \ldots \) with the same distribution as \( X \), there exist normalizing constants \( a_n \in \mathbb{R} \) and \( b_n > 0 \) such that
\[
\frac{X_1 + X_2 + \cdots + X_n - a_n}{b_n} \xrightarrow{d} \mathcal{N}(0, 1) \quad \text{as } n \to \infty.
\]
The central limit theorem implies that if \( \mathbb{E}[X^2] < \infty \), then \( X \) is in the domain of attraction of the normal distribution. However, if \( \mathbb{E}[X^2] = \infty \), additional conditions are needed to ensure this convergence.

\subsection*{Condition (a): Finite Second Moment}

Suppose that Condition (a) holds, i.e., \( \mathbb{E}[X^2] < \infty \).

If \( \mathbb{E}[X^2] < \infty \), then the variance \( \mathrm{Var}(X) = \mathbb{E}[X^2] - (\mathbb{E}[X])^2 \) is finite. This implies that we can apply the central limit theorem directly to the sum \( S_n = X_1 + X_2 + \cdots + X_n \) of \( n \) i.i.d.\ copies of \( X \). By the central limit theorem, there exist constants \( a_n = n \mathbb{E}[X] \) and \( b_n = \sqrt{n \, \mathrm{Var}(X)} \) such that
\[
\frac{S_n - a_n}{b_n} \xrightarrow{d} \mathcal{N}(0, 1) \quad \text{as } n \to \infty.
\]
Thus, under Condition (a), \( X \) lies in the domain of attraction of the normal distribution.

This proves one direction of the corollary: if \( \mathbb{E}[X^2] < \infty \), then \( X \) is in the domain of attraction of a normal distribution.

\subsection*{Condition (b): Infinite Second Moment and Tail Condition}

Now, suppose Condition (b) holds, i.e., \( \mathbb{E}[X^2] = \infty \) and
\[
G(x) = P(|X| > x) = o \left( x^{-2} \int_{|y| \leq x} y^2 \, dF(y) \right) \quad \text{as } x \to \infty.
\]
Our goal is to show that this tail condition (Equation (2.13)) implies that \( X \) still lies in the domain of attraction of the normal distribution, despite having infinite variance.

When \( \mathbb{E}[X^2] = \infty \), the variance of \( X \) does not exist, so we cannot apply the standard central limit theorem directly. Instead, we need to ensure that the contribution of large values of \( |X| \) to the distribution is controlled in such a way that the normalized sums \( S_n = X_1 + X_2 + \cdots + X_n \) still converge in distribution to a normal distribution.

\paragraph{Interpretation of Condition (2.13):}
Condition (2.13) implies that the tail probabilities \( P(|X| > x) \) decay faster than the rate \( x^{-2} \int_{|y| \leq x} y^2 \, dF(y) \) as \( x \to \infty \). Intuitively, this means that the probability of \( X \) taking large values is sufficiently small, so that no single term in the sum \( S_n = X_1 + X_2 + \cdots + X_n \) dominates the distribution of the sum.

\paragraph{Verification of the Tail Condition:}
To formally verify that Condition (2.13) implies convergence to a normal distribution, we can examine the truncated second moment of \( X \). Define the truncated second moment by
\[
\mathbb{E}[X^2; |X| \leq x] = \int_{|y| \leq x} y^2 \, dF(y).
\]
Condition (2.13) implies that for large \( x \), the tail probability \( P(|X| > x) \) is small enough that the sum \( S_n \) (even without finite variance) behaves in a manner that approximates normal convergence when normalized appropriately.

More specifically, since the probability mass in the tails is controlled, the central limit behavior for the truncated sums still leads to a normal limit. This result aligns with the broader theory of domains of attraction, where tail behavior often determines the limiting distribution when the variance is infinite.

Thus, Condition (b) ensures that \( X \) lies in the domain of attraction of a normal distribution, even though \( \mathbb{E}[X^2] = \infty \).


 \cite{r1,r2,r3,r4,r5,r6,r7,r8}
	\end{solve}
	% =================================================
	
	% \newpage
	
	% \vfill
	
	\bibliographystyle{apalike}
	\bibliography{references}
\end{document}