	
\input{preamble}
\usepackage{titlesec}
\usepackage[many]{tcolorbox}
\usepackage{graphics}
\usepackage{float}
\usepackage{multirow}
\usepackage{float}
\usepackage{caption}
\usepackage{amssymb}
\usepackage{amsmath}
\usepackage{lineno}
\usepackage{algorithm}
\usepackage{algorithmic}
\usepackage{multirow}
\usepackage{graphics}
\usepackage{bm}
\usepackage{float}
\usepackage{listings}
\usepackage{color}
\usepackage{graphicx}



% Adjust spacing after the chapter title
% \titlespacing*{<command>}{<left>}{<before-sep>}{<after-sep>}
\titlespacing*{\chapter}{0cm}{-2.0cm}{0.50cm}
\titlespacing*{\section}{0cm}{0.50cm}{0.25cm}

% Indent 
\setlength{\parindent}{0pt}
\setlength{\parskip}{1ex}

% --- Theorems, lemma, corollary, postulate, definition ---
% \numberwithin{equation}{section}


\newtcbtheorem[]{problem}{Problem}%
    {enhanced,
    colback = black!5, %white,
    colbacktitle = black!5,
    coltitle = black,
    boxrule = 0pt,
    frame hidden,
    borderline west = {0.5mm}{0.0mm}{black},
    fonttitle = \bfseries\sffamily,
    breakable,
    before skip = 3ex,
    after skip = 3ex
}{problem}

\newtcbtheorem[]{solve}{Solve}%
{enhanced,
	colback = black!5, %white,
	colbacktitle = black,
	coltitle=white,
	boxrule = 0pt,
	frame hidden,
	borderline west = {0.5mm}{0.0mm}{black},
	fonttitle = \bfseries\sffamily,
	breakable,
	before skip = 3ex,
	after skip = 3ex
}{solve}

\tcbuselibrary{skins, breakable}
\input{commands}


\begin{document}
		\begin{Large}
		\textsf{\textbf{Peter and Paul Distribution}}\\
		Section 6 - Home Work 3
	\end{Large}
	
	\vspace{1ex}
	
	\textsf{\textbf{Student:}} \text{Mehrab Atighi}, \href{mailto:mehrab.atighi@gmail.com}{\texttt{mehrab.atighi@gmail.com}}\\
	\textsf{\textbf{Lecturer:}} \text{Mohammad Zokaei}, \href{mailto:Zokaei@sbu.ac.ir}{\texttt{Zokaei@sbu.ac.ir}}
	
	
	\vspace{2ex}
	
	\begin{problem}{}{problem-label}
		Peter tosses a fair coin until it lands on heads for the first time. If this happens at trial \( k \), Peter receives \( 2^k \) Roubles from Paul. The distribution function (df) of Peter's gain is given by:
		
		\[
		F(x) = \sum_{k: 2^k \leq x} 2^{-k}, \quad x \geq 0
		\]
		\cite{Embrechts.etal1997}:
	\end{problem}
	
	\begin{solve}{}{solve-label}

We aim to show why the fraction \( \frac{F(2^k - 1)}{F(2^k)} \) equals 2.

1. \textit{CDF Definition:}
\[
F(2^k) = \sum_{j=1}^{k} 2^{-j}
\]

2. \textit{CDF at \(2^k - 1\):}
\[
F(2^k - 1) = \sum_{j=1}^{k-1} 2^{-j}
\]

3. \textit{Fraction Calculation:}
\[
\frac{F(2^k - 1)}{F(2^k)} = \frac{\sum_{j=1}^{k-1} 2^{-j}}{\sum_{j=1}^{k} 2^{-j}}
\]

4. \textit{Sum of a Geometric Series:}
The sum of a geometric series \( \sum_{j=0}^{k-1} ar^j \) is given by:
\[
\sum_{j=0}^{k-1} ar^j = a \frac{1-r^k}{1-r}
\]
For our series, \( a = 2^{-1} \) and \( r = 2^{-1} \), so we have:
\[
\sum_{j=1}^{k} 2^{-j} = \frac{1 - 2^{-k}}{1 - 2^{-1}} = 1 - 2^{-k}
\]

5. \textit{Simplifying the Fraction:}
\[
\frac{F(2^k - 1)}{F(2^k)} = \frac{1 - 2^{-(k-1)}}{1 - 2^{-k}}
\]

6. \textit{Simplifying the expression:}
\[
\frac{1 - 2^{-(k-1)}}{1 - 2^{-k}} = \frac{1 - \frac{1}{2^{k-1}}}{1 - \frac{1}{2^k}} = 2
\]

Therefore, the fraction \( \frac{F(2^k - 1)}{F(2^k)} \) equals 2.\\
\cite{r1,r2,r3,r4}
	\end{solve}
	% =================================================
	
	% \newpage
	
	% \vfill
	
	\bibliographystyle{apalike}
	\bibliography{references}
\end{document}