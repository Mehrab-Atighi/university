	
\documentclass[letterpaper, 11pt]{extarticle}
% \usepackage{fontspec}

% ==================================================

% document parameters
% \usepackage[spanish, mexico, es-lcroman]{babel}
\usepackage[english]{babel}
\usepackage[margin = 1in]{geometry}

% ==================================================

% Packages for math
\usepackage{mathrsfs}
\usepackage{amsfonts}
\usepackage{amsmath}
\usepackage{amsthm}
\usepackage{amssymb}
\usepackage{physics}
\usepackage{dsfont}
\usepackage{esint}

% ==================================================

% Packages for writing
\usepackage{enumerate}
\usepackage[shortlabels]{enumitem}
\usepackage{framed}
\usepackage{csquotes}

% ==================================================

% Miscellaneous packages
\usepackage{float}
\usepackage{tabularx}
\usepackage{xcolor}
\usepackage{multicol}
\usepackage{subcaption}
\usepackage{caption}
\captionsetup{format = hang, margin = 10pt, font = small, labelfont = bf}

% Citation
\usepackage[round, authoryear]{natbib}

% Hyperlinks setup
\usepackage{hyperref}
\definecolor{links}{rgb}{0.36,0.54,0.66}
\hypersetup{
   colorlinks = true,
    linkcolor = black,
     urlcolor = blue,
    citecolor = blue,
    filecolor = blue,
    pdfauthor = {Author},
     pdftitle = {Title},
   pdfsubject = {subject},
  pdfkeywords = {one, two},
  pdfproducer = {LaTeX},
   pdfcreator = {pdfLaTeX},
   }
\usepackage{titlesec}
\usepackage[many]{tcolorbox}

% Adjust spacing after the chapter title
% \titlespacing*{<command>}{<left>}{<before-sep>}{<after-sep>}
\titlespacing*{\chapter}{0cm}{-2.0cm}{0.50cm}
\titlespacing*{\section}{0cm}{0.50cm}{0.25cm}

% Indent 
\setlength{\parindent}{0pt}
\setlength{\parskip}{1ex}

% --- Theorems, lemma, corollary, postulate, definition ---
% \numberwithin{equation}{section}


\newtcbtheorem[]{problem}{Problem}%
    {enhanced,
    colback = black!5, %white,
    colbacktitle = black!5,
    coltitle = black,
    boxrule = 0pt,
    frame hidden,
    borderline west = {0.5mm}{0.0mm}{black},
    fonttitle = \bfseries\sffamily,
    breakable,
    before skip = 3ex,
    after skip = 3ex
}{problem}

\tcbuselibrary{skins, breakable}
%%% Commands defined by me
%   Euler constant
\newcommand{\eu}{\mathrm{e}}

%   Imaginary unit
\newcommand{\im}{\mathrm{i}}

%   Degrees
\newcommand{\grado}{\,^{\circ}}

%%% Linear Algebra

% Transpose
\newcommand{\transpose}[1]{{#1}^{\mathsf{T}}}

%%% Calculus
%   Integral from - infinity to infinity 
\newcommand{\Int}{\int\limits_{-\infty}^{\infty}}

%   Indefinite integral
\newcommand{\rint}[2]{\int{#1}\dd{#2}}

%   Definite integral
\newcommand{\Rint}[4]{\int\limits_{#1}^{#2}{#3}\dd{#4}}


% Serif bold text
\newcommand{\tsb}[1]{\textsf{\textbf{#1}}}

% Separation line
\newcommand{\linea}{\textcolor{gray!60}{\rule{\linewidth}{0.2pt}}}

% Bigger version of a \cdot to denote dot product
\makeatletter
\newcommand*\bigcdot{\mathpalette\bigcdot@{.5}}
\newcommand*\bigcdot@[2]{\mathbin{\vcenter{\hbox{\scalebox{#2}{$\m@th#1\bullet$}}}}}
\makeatother

% My Hamiltonian prefered notation
\newcommand{\Ham}{\hat{\mathcal{H}}}

%% Pre-existing commands redefined by me
% Trace of a matrix 
\renewcommand{\Tr}{\mathrm{Tr}}



\begin{document}
		\begin{Large}
		\textsf{\textbf{General claim arrival process proof}}\\
		Section 6 - Home Work 1
	\end{Large}
	
	\vspace{1ex}
	
	\textsf{\textbf{Student:}} \text{Mehrab Atighi}, \href{mailto:mehrab.atighi@gmail.com}{\texttt{mehrab.atighi@gmail.com}}\\
	\textsf{\textbf{Lecturer:}} \text{Mohammad Zokaei}, \href{mailto:Zokaei@sbu.ac.ir}{\texttt{Zokaei@sbu.ac.ir}}
	
	
	\vspace{2ex}
	
	\begin{problem}{}{problem-label}
		Consider equation (1.35) with $F \in S$. Fix $t > 0$, and suppose that the sequence $(p_t(n))$ satisfies:
		\[
		\sum_{n=0}^{\infty} (1 + \epsilon)^n p_t(n) < \infty
		\]
		for some $\epsilon > 0$. We need to show that $G_t \in S$ and that
		\[
		\overline{G_t}(x) \sim EN(t) \overline{F}(x), \quad x \to \infty.
		\]\cite{Embrechts.etal1997}:
	\end{problem}
	
	\begin{solve}{}{solve-label}
\textbf{Cramér-Lundberg Theorem for Large Claims I:}

\textit{Theorem (Cramér-Lundberg I):} In a risk process where the claim size distribution $F$ has a subexponential tail, the ruin probability satisfies:
\[
\psi(u) \sim \rho^{-1} \overline{F_I}(u) \text{ as } u \to \infty,
\]
where $\rho$ is the adjustment coefficient, and $\overline{F_I}(u)$ represents the tail distribution of the integrated claim size.

\textbf{Proof Using Cramér-Lundberg Theorem}

1. \textit{Subexponential Assumption:}
Given $F \in S$, we know that $F$ is subexponential. This implies for all $t > 0$:
\[
\overline{F*F}(x) \sim 2\overline{F}(x).
\]

2. \textit{Transformation of $G_t$:}
Define $G_t$ through the mixture transformation with \textbf{weights $p_t(n)$}:
\[
\overline{G_t}(x) = \sum_{n=0}^{\infty} p_t(n) \overline{F^{*n}}(x).
\]

3. \textit{Applying Cramér-Lundberg:}
From the Cramér-Lundberg theorem, for large claims, the tail distribution of the aggregate claims can be approximated by:
\[
\psi(u) \sim \rho^{-1} \overline{F_I}(u) \text{ as } u \to \infty.
\]

4. \textit{Simplifying the Asymptotic Behavior:}
Using the properties of subexponential distributions and the decay condition on $p_t(n)$:
\[
\overline{G_t}(x) = \sum_{n=0}^{\infty} p_t(n) \overline{F^{*n}}(x) \sim EN(t) \overline{F}(x).
\]

5. \textit{Result:}
Therefore, by the property of subexponentiality:
\[
\overline{G_t}(x) \sim EN(t) \overline{F}(x), \quad x \to \infty.
\]

This shows that $G_t \in S$ and completes the proof of Theorem 1.3.9. \\\cite{r1,r2,r3}
	\end{solve}
	% =================================================
	
	% \newpage
	
	% \vfill
	
	\bibliographystyle{apalike}
	\bibliography{references}
\end{document}