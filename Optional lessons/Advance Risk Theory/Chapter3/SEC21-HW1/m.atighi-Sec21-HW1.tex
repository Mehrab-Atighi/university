	
\documentclass[letterpaper, 11pt]{extarticle}
% \usepackage{fontspec}

% ==================================================

% document parameters
% \usepackage[spanish, mexico, es-lcroman]{babel}
\usepackage[english]{babel}
\usepackage[margin = 1in]{geometry}

% ==================================================

% Packages for math
\usepackage{mathrsfs}
\usepackage{amsfonts}
\usepackage{amsmath}
\usepackage{amsthm}
\usepackage{amssymb}
\usepackage{physics}
\usepackage{dsfont}
\usepackage{esint}

% ==================================================

% Packages for writing
\usepackage{enumerate}
\usepackage[shortlabels]{enumitem}
\usepackage{framed}
\usepackage{csquotes}

% ==================================================

% Miscellaneous packages
\usepackage{float}
\usepackage{tabularx}
\usepackage{xcolor}
\usepackage{multicol}
\usepackage{subcaption}
\usepackage{caption}
\captionsetup{format = hang, margin = 10pt, font = small, labelfont = bf}

% Citation
\usepackage[round, authoryear]{natbib}

% Hyperlinks setup
\usepackage{hyperref}
\definecolor{links}{rgb}{0.36,0.54,0.66}
\hypersetup{
   colorlinks = true,
    linkcolor = black,
     urlcolor = blue,
    citecolor = blue,
    filecolor = blue,
    pdfauthor = {Author},
     pdftitle = {Title},
   pdfsubject = {subject},
  pdfkeywords = {one, two},
  pdfproducer = {LaTeX},
   pdfcreator = {pdfLaTeX},
   }
\usepackage{titlesec}
\usepackage[many]{tcolorbox}

% Adjust spacing after the chapter title
% \titlespacing*{<command>}{<left>}{<before-sep>}{<after-sep>}
\titlespacing*{\chapter}{0cm}{-2.0cm}{0.50cm}
\titlespacing*{\section}{0cm}{0.50cm}{0.25cm}

% Indent 
\setlength{\parindent}{0pt}
\setlength{\parskip}{1ex}

% --- Theorems, lemma, corollary, postulate, definition ---
% \numberwithin{equation}{section}


\newtcbtheorem[]{problem}{Problem}%
    {enhanced,
    colback = black!5, %white,
    colbacktitle = black!5,
    coltitle = black,
    boxrule = 0pt,
    frame hidden,
    borderline west = {0.5mm}{0.0mm}{black},
    fonttitle = \bfseries\sffamily,
    breakable,
    before skip = 3ex,
    after skip = 3ex
}{problem}

\tcbuselibrary{skins, breakable}
%%% Commands defined by me
%   Euler constant
\newcommand{\eu}{\mathrm{e}}

%   Imaginary unit
\newcommand{\im}{\mathrm{i}}

%   Degrees
\newcommand{\grado}{\,^{\circ}}

%%% Linear Algebra

% Transpose
\newcommand{\transpose}[1]{{#1}^{\mathsf{T}}}

%%% Calculus
%   Integral from - infinity to infinity 
\newcommand{\Int}{\int\limits_{-\infty}^{\infty}}

%   Indefinite integral
\newcommand{\rint}[2]{\int{#1}\dd{#2}}

%   Definite integral
\newcommand{\Rint}[4]{\int\limits_{#1}^{#2}{#3}\dd{#4}}


% Serif bold text
\newcommand{\tsb}[1]{\textsf{\textbf{#1}}}

% Separation line
\newcommand{\linea}{\textcolor{gray!60}{\rule{\linewidth}{0.2pt}}}

% Bigger version of a \cdot to denote dot product
\makeatletter
\newcommand*\bigcdot{\mathpalette\bigcdot@{.5}}
\newcommand*\bigcdot@[2]{\mathbin{\vcenter{\hbox{\scalebox{#2}{$\m@th#1\bullet$}}}}}
\makeatother

% My Hamiltonian prefered notation
\newcommand{\Ham}{\hat{\mathcal{H}}}

%% Pre-existing commands redefined by me
% Trace of a matrix 
\renewcommand{\Tr}{\mathrm{Tr}}



\begin{document}
		\begin{Large}
		\textsf{\textbf{Find norming constant of normal distribution}}\\
		Section 21 - Home Work 1
	\end{Large}
	
	\vspace{1ex}
	
	\textsf{\textbf{Student:}} \text{Mehrab Atighi}, \href{mailto:mehrab.atighi@gmail.com}{\texttt{mehrab.atighi@gmail.com}}\\
	\textsf{\textbf{Lecturer:}} \text{Mohammad Zokaei}, \href{mailto:Zokaei@sbu.ac.ir}{\texttt{Zokaei@sbu.ac.ir}}
	
	
	\vspace{2ex}
	
	\begin{problem}{}{problem-label}
				\textbf{According to the Example 3.3.29 find norming constant of normal distribution.}
				\cite{Embrechts.etal1997}
	\end{problem}
	
	\begin{solve}{}{solve-label}
\subsection*{1. Von Mises Function and Condition (3.25)}

We denote by $\Phi$ the distribution function (df) and by $\varphi$ the density of the standard normal distribution. First, we need to show that $\Phi$ is a von Mises function and satisfies condition (3.25).

An application of l'Hospital's rule to 

\[ 
\frac{\overline{\Phi}(x)}{x^{-1}\varphi(x)} 
\]

yields Mill's ratio: 

\[ 
\overline{\Phi}(x) \sim \frac{\varphi(x)}{x} 
\]

This implies:

\[ 
\varphi'(x) = -x \varphi(x) < 0 
\]

and

\[ 
\lim_{x \to \infty} \frac{\overline{\Phi}(x) \varphi'(x)}{\varphi^2(x)} = -1. 
\]

Thus, $\Phi \in \text{MDA}(\Lambda)$ by Example 3.3.23 and Proposition 3.3.25.

\subsection*{2. Calculation of Mill's Ratio}

The tail probability (survival function) is given by:

\[ 
\overline{\Phi}(x) = 1 - \Phi(x). 
\]

Mill's ratio is expressed as:

\[ 
R(x) = \frac{\overline{\Phi}(x)}{\varphi(x)}. 
\]

For large values of $x$, the Mill's ratio can be approximated as:

\[ 
\overline{\Phi}(x) \sim \frac{\varphi(x)}{x}. 
\]

\subsubsection*{Derivation of Mill's Ratio}

\begin{enumerate}
	\item \textbf{Expression for the Standard Normal Density Function:}
	The probability density function for the standard normal distribution is:
	
	\[ 
	\varphi(x) = \frac{1}{\sqrt{2\pi}} e^{-x^2/2}. 
	\]
	
	\item \textbf{Applying l'Hospital's Rule:}
	To find the asymptotic behavior of $\overline{\Phi}(x)$ for large $x$, we evaluate the ratio \(\frac{\overline{\Phi}(x)}{\varphi(x)}\) using l'Hospital's rule.
	
	Since $\overline{\Phi}(x) = 1 - \Phi(x)$ and the derivative of $\Phi(x)$ is $\varphi(x)$:
	
	\[ 
	\Phi'(x) = \varphi(x),
	\]
	
	and the derivative of $\varphi(x)$ is:
	
	\[ 
	\varphi'(x) = -x \varphi(x). 
	\]
	
	Using these derivatives:
	
	\[ 
	\lim_{x \to \infty} \frac{\overline{\Phi}(x)}{\varphi(x)} = \lim_{x \to \infty} \frac{-\Phi'(x)}{-\varphi'(x)} = \lim_{x \to \infty} \frac{\varphi(x)}{x \varphi(x)} = \lim_{x \to \infty} \frac{1}{x}. 
	\]
	
	Thus, for large $x$:
	
	\[ 
	\overline{\Phi}(x) \sim \frac{\varphi(x)}{x}. 
	\]
	
\end{enumerate}

\subsection*{3. Solving for $d_n$ and $c_n$}

\subsubsection*{Deriving $d_n$}

Using Proposition 3.3.28, the norming constant $d_n$ satisfies:

\[ 
-\ln G(d_n) = \ln n. 
\]

For the standard normal distribution, $G(d_n)$ represents the upper tail probability:

\[ 
\overline{\Phi}(d_n) = \frac{1}{\sqrt{2\pi}} \int_{d_n}^{\infty} e^{-t^2/2} \, dt. 
\]

Using the asymptotic behavior of $\overline{\Phi}(x)$:

\[ 
\overline{\Phi}(x) \sim \frac{1}{\sqrt{2\pi}} \frac{e^{-x^2/2}}{x}, 
\]

we have:

\[ 
-\ln \overline{\Phi}(d_n) = \ln n 
\]

i.e.,

\[ 
\ln \left( \frac{1}{\sqrt{2\pi}} \frac{e^{-d_n^2/2}}{d_n} \right) = \ln n. 
\]

Expanding the logarithm:

\[ 
-\frac{d_n^2}{2} - \ln d_n - \frac{1}{2}\ln(2\pi) = \ln n. 
\]

Rearranging terms gives:

\[ 
\frac{1}{2} d_n^2 + \ln d_n + \frac{1}{2} \ln(2\pi) = \ln n. 
\]

Now, solving this equation for $d_n$, we use an asymptotic expansion. Assume $d_n$ is large, so the leading term is dominant:

1. The leading-order approximation is:

\[ 
\frac{1}{2} d_n^2 \approx \ln n \implies d_n \approx (2 \ln n)^{1/2}. 
\]

2. To refine this, substitute $d_n = (2 \ln n)^{1/2} + \delta$ and solve for $\delta$. Substituting into the equation:

\[ 
\frac{1}{2} \left((2 \ln n)^{1/2} + \delta\right)^2 + \ln \left((2 \ln n)^{1/2} + \delta\right) + \frac{1}{2} \ln(2\pi) = \ln n. 
\]

Expanding and keeping terms up to $\delta$:

\[ 
\frac{1}{2} (2 \ln n) + \delta (2 \ln n)^{1/2} + \ln (2 \ln n)^{1/2} + \frac{1}{2}\ln(2\pi) \approx \ln n. 
\]

Solving for $\delta$:

\[ 
\delta \approx -\frac{\ln \ln n + \ln(4\pi)}{2(2 \ln n)^{1/2}}. 
\]

Thus, the refined approximation for $d_n$ is:

\[ 
d_n = (2 \ln n)^{1/2} - \frac{\ln \ln n + \ln(4\pi)}{2(2 \ln n)^{1/2}} + o((\ln n)^{-1/2}). 
\]

\subsubsection*{Deriving $c_n$}

To derive $c_n$, recall:

\[ 
R(x) = \frac{\overline{\Phi}(x)}{\varphi(x)} \sim \frac{1}{x}. 
\]

Thus:

\[ 
c_n = a(d_n) \sim \frac{1}{d_n}. 
\]

Substituting the expansion for $d_n$:

\[ 
c_n \sim \frac{1}{(2 \ln n)^{1/2}}. 
\]

This is the leading-order term for $c_n$. Higher-order corrections can be included if necessary, but typically the leading-order term suffices for most asymptotic analyses.

\subsection*{4. Summary of Results}

\begin{itemize}
	\item The norming constant $d_n$ is approximated as:
	\[ 
	d_n = (2 \ln n)^{1/2} - \frac{\ln \ln n + \ln(4\pi)}{2(2 \ln n)^{1/2}} + o((\ln n)^{-1/2}). 
	\]
	
	\item The norming constant $c_n$ is approximated as:
	\[ 
	c_n \sim \frac{1}{(2 \ln n)^{1/2}}. 
	\]
	
	\item These constants are crucial for understanding the asymptotic behavior of the standard normal distribution in extreme value theory.
\end{itemize}

 \cite{Embrechts.etal1997}
	\end{solve}
	% =================================================
	
	% \newpage
	
	% \vfill
	
	\bibliographystyle{apalike}
	\bibliography{references}
\end{document}