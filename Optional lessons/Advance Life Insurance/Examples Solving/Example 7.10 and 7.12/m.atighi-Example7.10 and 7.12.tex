\documentclass[letterpaper, 11pt]{extarticle}
% \usepackage{fontspec}

% ==================================================

% document parameters
% \usepackage[spanish, mexico, es-lcroman]{babel}
\usepackage[english]{babel}
\usepackage[margin = 1in]{geometry}

% ==================================================

% Packages for math
\usepackage{mathrsfs}
\usepackage{amsfonts}
\usepackage{amsmath}
\usepackage{amsthm}
\usepackage{amssymb}
\usepackage{physics}
\usepackage{dsfont}
\usepackage{esint}

% ==================================================

% Packages for writing
\usepackage{enumerate}
\usepackage[shortlabels]{enumitem}
\usepackage{framed}
\usepackage{csquotes}

% ==================================================

% Miscellaneous packages
\usepackage{float}
\usepackage{tabularx}
\usepackage{xcolor}
\usepackage{multicol}
\usepackage{subcaption}
\usepackage{caption}
\captionsetup{format = hang, margin = 10pt, font = small, labelfont = bf}

% Citation
\usepackage[round, authoryear]{natbib}

% Hyperlinks setup
\usepackage{hyperref}
\definecolor{links}{rgb}{0.36,0.54,0.66}
\hypersetup{
   colorlinks = true,
    linkcolor = black,
     urlcolor = blue,
    citecolor = blue,
    filecolor = blue,
    pdfauthor = {Author},
     pdftitle = {Title},
   pdfsubject = {subject},
  pdfkeywords = {one, two},
  pdfproducer = {LaTeX},
   pdfcreator = {pdfLaTeX},
   }
\usepackage{titlesec}
\usepackage[many]{tcolorbox}

% Adjust spacing after the chapter title
% \titlespacing*{<command>}{<left>}{<before-sep>}{<after-sep>}
\titlespacing*{\chapter}{0cm}{-2.0cm}{0.50cm}
\titlespacing*{\section}{0cm}{0.50cm}{0.25cm}

% Indent 
\setlength{\parindent}{0pt}
\setlength{\parskip}{1ex}

% --- Theorems, lemma, corollary, postulate, definition ---
% \numberwithin{equation}{section}


\newtcbtheorem[]{problem}{Problem}%
    {enhanced,
    colback = black!5, %white,
    colbacktitle = black!5,
    coltitle = black,
    boxrule = 0pt,
    frame hidden,
    borderline west = {0.5mm}{0.0mm}{black},
    fonttitle = \bfseries\sffamily,
    breakable,
    before skip = 3ex,
    after skip = 3ex
}{problem}

\tcbuselibrary{skins, breakable}
%%% Commands defined by me
%   Euler constant
\newcommand{\eu}{\mathrm{e}}

%   Imaginary unit
\newcommand{\im}{\mathrm{i}}

%   Degrees
\newcommand{\grado}{\,^{\circ}}

%%% Linear Algebra

% Transpose
\newcommand{\transpose}[1]{{#1}^{\mathsf{T}}}

%%% Calculus
%   Integral from - infinity to infinity 
\newcommand{\Int}{\int\limits_{-\infty}^{\infty}}

%   Indefinite integral
\newcommand{\rint}[2]{\int{#1}\dd{#2}}

%   Definite integral
\newcommand{\Rint}[4]{\int\limits_{#1}^{#2}{#3}\dd{#4}}


% Serif bold text
\newcommand{\tsb}[1]{\textsf{\textbf{#1}}}

% Separation line
\newcommand{\linea}{\textcolor{gray!60}{\rule{\linewidth}{0.2pt}}}

% Bigger version of a \cdot to denote dot product
\makeatletter
\newcommand*\bigcdot{\mathpalette\bigcdot@{.5}}
\newcommand*\bigcdot@[2]{\mathbin{\vcenter{\hbox{\scalebox{#2}{$\m@th#1\bullet$}}}}}
\makeatother

% My Hamiltonian prefered notation
\newcommand{\Ham}{\hat{\mathcal{H}}}

%% Pre-existing commands redefined by me
% Trace of a matrix 
\renewcommand{\Tr}{\mathrm{Tr}}

\usepackage{actuarialsymbol}

\begin{document}
	\begin{Large}
		\textsf{\textbf{Gross Premium Policy Value Calculations}}\\
		Example 7.10 and 7.12
	\end{Large}
	
	\vspace{1ex}
	
	\textsf{\textbf{Student:}} \text{Mehrab Atighi}, \href{mailto:mehrab.atighi@gmail.com}{\texttt{mehrab.atighi@gmail.com}}\\
	\textsf{\textbf{Teacher:}} \text{Shirin Shoaee}, \href{mailto:sh_shoaee@sbu.ac.ir}{\texttt{sh\_shoaee@sbu.ac.ir}}
	
	\vspace{2ex}
	
	\begin{problem}{}{Example 7.10}
		A life aged 50 purchases a 10-year term insurance with sum insured \$500,000 payable at the end of the month of death. Level monthly premiums, each of amount $P = 460$, are payable for at most five years.
		
		Calculate the (gross premium) policy values at durations 2.75, 3, and 6.5 years using the following basis:
		\begin{itemize}
			\item \textbf{Survival model:} Standard Select Survival Model
			\item \textbf{Interest:} 5\% per year
			\item \textbf{Expenses:} 10\% of each gross premium
		\end{itemize}
	\end{problem}
	\begin{problem}{}{Example 7.11}
		According to the problem1 (Example 7.10) Calculate the (gross premium) policy value at the duration 2.8. \\
		\textbf{Attention that t = 2.8 is between premium dates }
	\end{problem}
	
	\begin{solve}{}{Example 7.10 solution}
		\section*{Solution}
		
		\subsection*{Step 1: Define the Random Loss Variable}
		The random loss variable $L_t$ at time $t$ is defined as:
		\[
		L_t = B_t - P_t - E_t,
		\]
		where:
		\begin{itemize}
			\item $B_t$: the benefits variable at time $t$,
			\item $P_t$: the premiums variable  at time $t$,
			\item $E_t$: the expenses variable at time $t$.
		\end{itemize}
		
		\subsection*{Step 2: Components of the Loss Variables}
		\subsubsection*{1. Benefits}
		The benefits are \$500,000, payable at the end of the month of death. The present value of benefits as time t is given by:
		\[
		B_t = 500,000 \cdot \sum_{k=1}^{n} v^{k/12} \cdot \,_{t+k/12}q_{x},
		\]
		where:
		\begin{itemize}
			\item $n = 120$ months (10 years),
			\item $v = (1+i)^{-1}$ is the monthly discount factor, $i = 0.05$ annually,
			\item $_{t+k/12}q_{x}$ is the probability of dying in the $(k/12)$th month.
		\end{itemize}
		
		\subsubsection*{2. Premiums}
		The monthly premium is \$460. The present value of premiums at time t is:
		\[
		P_t = \sum_{k=1}^{m} 4 * P \cdot v^{k/4} \cdot \,_{t+k/4}p_{x} = \sum_{k=1}^{m} 4 * 460 \cdot v^{k/4} \cdot \,_{t+k/4}p_{x}.
		\]
		where:
		\begin{itemize}
			\item $m = 20$ Quarters (5 years maximum of Quarterly premiums),
			\item $_{t+k/4}p_{x}$ is the probability of survival to the $(k/4)$th Quarter.
		\end{itemize}
		
		\subsubsection*{3. Expenses}
		Expenses are 10\% of each gross premium paid, so the monthly expense is:
		\[
		\text{Expense per premium payment} = 0.1 \cdot P = 46.
		\]
		The present value of expenses at time t is:
		\[
		E_t = \sum_{k=1}^{m} 4 * 0.1P \cdot v^{k/4} \cdot \,_{t+k/4}p_{x} = \sum_{k=1}^{m} 4 * 46 \cdot v^{k/4} \cdot \,_{t+k/4}p_{x}. 
		\]
		
		\subsection*{Step 3: Expected Present Value (EPV) of the Loss Random Variable}
		Using the survival model, the gross premium policy value at time $t$ is the expected value of the loss random variable:
		\[
		V_t = \text{EPV}(Z_t) - \text{EPV}(P_t) - \text{EPV}(E_t).
		\]
		\subsection*{Step 4: Components of the Expected Present Value (EPV) of the Loss Random Variable}
		\subsubsection*{1. Expected Present Value of Benefits (\(Z_t\))}
		
		The benefits are \$500,000, payable at the end of the month of death. The expected present value (EPV) of these benefits at time \(t\) is denoted by:
		\[
		EPV(B)_t = S \cdot \,  \ddot{A}_{\actuarialangle{x:10}}^{(12)} = 500,000 \cdot \,  \ddot{A}_{\actuarialangle{x:10}}^{(12)}.
		\]
		where:
		\begin{itemize}
			\item \(\ddot{A}_{\actuarialangle{x:10}}^{(12)}\): The expected present value of a 1-unit insurance benefit, payable at the end of the month of death, for a life aged \(x\), with payments made 12 times per year.
			\item The symbol \(\ddot{A}_{\actuarialangle{x:10}}^{(12)}\) incorporates the monthly interest rate and survival probabilities, as follows:
			\[
			\ddot{A}_{\actuarialangle{x:n}}^{(12)} = \sum_{k=1}^{n} v^{k/12} \cdot \, _{k/12}q_x.
			\]
			\item \(v = (1+i)^{-1}\): The annual discount factor, with \(i = 0.05\).
			\item \(n = 120\): The total number of months for the 10-year term insurance.
			\item \(_{k/12}q_x\): The probability that the life dies in the \((k/12)\)th month.
		\end{itemize}
		
		\subsubsection*{2. Expected Present Value of Premiums (\(P_t\))}
		
		The premiums are \$460, payable Quarterly for a maximum of 5 years (20 Quarter). The expected present value (EPV) of these premiums at time \(t\) is denoted by:
		\[
		EPV(P)_t = 4 * P \cdot \, \ddot{a}_{\actuarialangle{x:5}}^{(4)} = 4 * 460 \cdot \, \ddot{a}_{\actuarialangle{x:5}}^{(4)}.
		\]
		where:
		\begin{itemize}
			\item \(\ddot{a}_{\actuarialangle{x:n}}^{(12)}\): The expected present value of a 1-unit premium paid Quarterly in advance for a life aged \(x\), for a maximum of n years, with payments made 4 times per year.
			\item The symbol \(\ddot{a}_{\actuarialangle{x:n}}^{(4)}\) incorporates the monthly interest rate and survival probabilities, as follows:
			\[
			\ddot{a}_{\actuarialangle{x:n}}^{(4)} = \sum_{k=0}^{nm-1} v^{k/4} \cdot \, _{k/4}p_x,
			\]
			where:
			\begin{itemize}
				\item \(v = (1+i)^{-1}\): The annual discount factor, with \(i = 0.05\),
				\item \(nm = 5*4 = 20\): The total number of Quarter premiums (5 years),
				\item \(_{k/4}p_x\): The probability that the life survives to the \((k/4)\)th month.
			\end{itemize}
		\end{itemize}
		
		\subsubsection*{3. Expected Present Value of Expenses (\(E_t\))}
		
		The expenses are 10\% of each premium, i.e., \$46, payable Quarterly for a maximum of 5 years (20 Quarte). The expected present value (EPV) of these expenses at time \(t\) is denoted by:
		\[
		EPV(E)_t = 4 * 0.1P \cdot \, \ddot{a}_{\actuarialangle{x:5}}^{(4)} = 4 * 46 \cdot \, \ddot{a}_{\actuarialangle{x:5}}^{(4)}.
		\]
		where:
		\begin{itemize}
			\item \(\ddot{a}_{\actuarialangle{x:n}}^{(4)}\): The expected present value of a 1-unit expense paid Quarterly in advance for a life aged \(x\), for a maximum of n=6 years, with payments made 4 times per year.
			\item The symbol \(\ddot{a}_{x}^{(4)}\) is the same as used in the premium calculation:
			\[
			\ddot{a}_{\actuarialangle{x:n}}^{(4)} = \sum_{k=0}^{nm-1} v^{k/4} \cdot \, _{k/4}p_x,
			\]
			where:
			\begin{itemize}
				\item \(v = (1+i)^{-1}\): The annual discount factor, with \(i = 0.05\),
				\item \(nm = 5*4 = 20\): The total number of monthly expense payments (5 years),
				\item \(_{k/4}p_x\): The probability that the life survives to the \((k/4)\)th month.
			\end{itemize}
		\end{itemize}
		
		
		\subsection*{Step 5: Policy Value Calculation at Times \(t = 2.75\), \(t = 3\), and \(t = 6.5\)}
		
		The policy value at any time \(t\) is given by the general formula:
		\[
		tV_x + EPV_t(\text{future premiums}) = EPV_t(\text{future benefits}) + EPV_t(\text{future expenses}),
		\]
		where:
		\begin{itemize}
			\item \(tV_x\): The policy value at time \(t\) for a life aged \(x+t\).
			\item \(EPV_t(\text{future premiums})\): The expected present value of premiums still to be paid after time \(t\).
			\item \(EPV_t(\text{future benefits})\): The expected present value of benefits payable after time \(t\).
			\item \(EPV_t(\text{future expenses})\): The expected present value of expenses after time \(t\).
		\end{itemize}
		
		To calculate the policy value at times \(t = 2.75\), \(t = 3\), and \(t = 6.5\):
		The policy value at  \(t = 2.75 \) is given by the general formula:
		\[
		2.75V_{50} = EPV_{2.75}(\text{B}) + EPV_{2.75}(\text{E}) -  EPV_{2.75}(\text{P})
		\]
		Now we will place the values that we calculated above for t = 2.75.
		\[
		2.75V_{50} = 500,000 \cdot \,  \ddot{A}_{\actuarialangle{52.75:7.25}}^{(12)} + 4 * 0.1P \cdot \, \ddot{a}_{\actuarialangle{52.75:2.25}}^{(4)} - 4 * P \cdot \, \ddot{a}_{\actuarialangle{52.75:2.25}}^{(4)}
		\]
		
		$$= 
		500,000 \cdot \,  \ddot{A}_{\actuarialangle{52.75:7.25}}^{(12)} - 4 * 0.9P \cdot \, \ddot{a}_{\actuarialangle{52.75:2.25}}^{(4)} = 3091.02\$ $$
		
	
		Similarly for t = 3.
		\[
		3V_{50} = 500,000 \cdot \,  \ddot{A}_{\actuarialangle{53:7}}^{(12)} + 4 * 0.1P \cdot \, \ddot{a}_{\actuarialangle{53:2}}^{(4)} - 4 * P \cdot \, \ddot{a}_{\actuarialangle{53:2}}^{(4)}
		\]
		
		$$= 
		500,000 \cdot \,  \ddot{A}_{\actuarialangle{53:7}}^{(12)} - 4 * 0.9P \cdot \, \ddot{a}_{\actuarialangle{53:2}}^{(4)} = 3357.94\$ $$
		And finally similarly for t = 3.
		we should note that there is no premium in t = 6 so we dont have premiums and expenses!.
		\[
		6.5V_{50} = 500,000 \cdot \,  \ddot{A}_{\actuarialangle{56.5:3.5}}^{(12)} 
		\]
		
		$$= 
		500,000 \cdot \,  \ddot{A}_{\actuarialangle{56.5:3.5}}^{(12)} = 4265.63\$ $$
		
	
		
	\end{solve}
	\begin{solve}{}{Example 7.12 solution}
		\section*{Solution}
		According to the Above information and the note that valuation time (2.8) is between premium dates.\\
		
		we have \textbf{two method for solve} this example.
	\begin{enumerate}
		\item Using Recursions formula:\\
		we should split the EPV of future cash flows into the part up to the next premium date or benefit date and the parts
		after that date.\\
		The EPV of future benefits is 
		\begin{align*} S \nu^{0.033} \frac{0.033 q_{52.8}}{0.033} A_{\actuarialangle{52.833:7.167}}^{(12) 1} &= 500\,000 \nu^{0.033} \frac{0.033 q_{52.8}}{0.033} A_{\actuarialangle{52.833:7.167}}^{(12) 1} \\ &= 6614.75. 
		\end{align*}
		
		The EPV of future premiums less expenses is \[ 0.9 \times 4 P \nu^{0.2} 0.2 P_{52.8} \ddot{a}_{\actuarialangle{53:2}}^{(4)} = 3138.59.\] In each of the above, we have used an exact calculation. Combining these, \[ 2.8 V = 6614.75 - 3138.59 = 3476.16. \] An alternative approach is to use a form of recursion from time \( t = 2.8 \) up to the next benefit date, time \( t = 2.833 \), where we can use formula (7.9). 
		We can derive recursive formulae for policy values for policies with cash flows at discrete times other than annually. For a policy with premiums of \(P\) every \(1/m\) years, premium expenses of \(e_t\) at time \(t\), and sum insured \(S\) payable at the end of the \(1/m\)th year of death, then following exactly the reasoning for the annual case, for \(t = 0, \frac{1}{m}, \frac{2}{m}, \frac{3}{m}, \ldots\), we have 
		
		\begin{align*}
			\left(_tV + P - e_t\right) (1 + i)^{\frac{1}{m}} &= _\frac{1}{m} q_{[x]+t} S + _\frac{1}{m}P_{[x]+t} \left(_{t + \frac{1}{m}}V\right) \\
			&= _{t +\frac{1}{m}}V + _{\frac{1}{m}}q_{[x]+t} \left(S - _{t + \frac{1}{m} }V\right).
		\end{align*}
		On the left-hand side, we have the policy value brought forward, with the addition of the premium less premium expenses, and with interest added at the end of the \(1/m\) years. On the right-hand side, we have the expected cost of death claims arising during the period, plus the expected cost of the policy value to be carried forward to the next period. In expectation, everything is in balance.\\
		So Accumulating \( 2.8 V \) for 0.033 of a year to provide \( S \) if death occurs before time 2.833 or \( 2.833 V \) if the policyholder survives to time 2.833, we have \[ 2.8 V (1 + i)^{0.033} = 0.033 q_{52.8} S + 0.033 P_{52.8} (2.833 V) \] giving \( 2.8 V = 3476.16 \), as before.
	\end{enumerate}
		
	\end{solve}
	\vspace{2ex}
	
	% =================================================
	
	% \newpage
	
	% \vfill
%	
%	\bibliographystyle{apalike}
%	\bibliography{references}
\end{document}
