	
\input{preamble}
\usepackage{titlesec}
\usepackage[many]{tcolorbox}
\usepackage{graphics}
\usepackage{float}
\usepackage{multirow}
\usepackage{float}
\usepackage{caption}
\usepackage{amssymb}
\usepackage{amsmath}
\usepackage{lineno}
\usepackage{algorithm}
\usepackage{algorithmic}
\usepackage{multirow}
\usepackage{graphics}
\usepackage{bm}
\usepackage{float}
\usepackage{listings}
\usepackage{color}
\usepackage{graphicx}



% Adjust spacing after the chapter title
% \titlespacing*{<command>}{<left>}{<before-sep>}{<after-sep>}
\titlespacing*{\chapter}{0cm}{-2.0cm}{0.50cm}
\titlespacing*{\section}{0cm}{0.50cm}{0.25cm}

% Indent 
\setlength{\parindent}{0pt}
\setlength{\parskip}{1ex}

% --- Theorems, lemma, corollary, postulate, definition ---
% \numberwithin{equation}{section}


\newtcbtheorem[]{problem}{Problem}%
    {enhanced,
    colback = black!5, %white,
    colbacktitle = black!5,
    coltitle = black,
    boxrule = 0pt,
    frame hidden,
    borderline west = {0.5mm}{0.0mm}{black},
    fonttitle = \bfseries\sffamily,
    breakable,
    before skip = 3ex,
    after skip = 3ex
}{problem}

\newtcbtheorem[]{solve}{Solve}%
{enhanced,
	colback = black!5, %white,
	colbacktitle = black,
	coltitle=white,
	boxrule = 0pt,
	frame hidden,
	borderline west = {0.5mm}{0.0mm}{black},
	fonttitle = \bfseries\sffamily,
	breakable,
	before skip = 3ex,
	after skip = 3ex
}{solve}

\tcbuselibrary{skins, breakable}
\input{commands}


\begin{document}
		\begin{Large}
		\textsf{\textbf{Relationship between the negative binomial distribution and the gamma}}\\
		Section 6 - Home Work 2
	\end{Large}
	
	\vspace{1ex}
	
	\textsf{\textbf{Student:}} \text{Mehrab Atighi}, \href{mailto:mehrab.atighi@gmail.com}{\texttt{mehrab.atighi@gmail.com}}\\
	\textsf{\textbf{Lecturer:}} \text{Mohammad Zokaei}, \href{mailto:Zokaei@sbu.ac.ir}{\texttt{Zokaei@sbu.ac.ir}}
	
	
	\vspace{2ex}
	
	\begin{problem}{}{problem-label}
		Consider equation (1.35) with $F \in S$. Fix $t > 0$, and suppose that the sequence $(p_t(n))$ satisfies:
		\[
		\sum_{n=0}^{\infty} (1 + \epsilon)^n p_t(n) < \infty
		\]
		for some $\epsilon > 0$. We need to show that $G_t \in S$ and that
		\[
		\overline{G_t}(x) \sim EN(t) \overline{F}(x), \quad x \to \infty.
		\]\cite{Embrechts.etal1997}:
	\end{problem}
	
	\begin{solve}{}{solve-label}
The negative binomial distribution describes the probability of having \(k\) failures before achieving \(r\) successes in a sequence of independent and identically distributed Bernoulli trials. This distribution can be elegantly expressed using the gamma function.

\section*{Negative Binomial Distribution}

The probability mass function (PMF) of the negative binomial distribution for a random variable \(X\) is given by:

\[
P(X = k) = \binom{r+k-1}{k} (1-p)^r p^k
\]

where:
\begin{itemize}
	\item \(r\) is the number of successes,
	\item \(p\) is the probability of success,
	\item \(k\) is the number of failures.
\end{itemize}

\section*{Using the Gamma Function}

Using the gamma function, which is defined as \(\Gamma(n) = (n-1)!\), the PMF can be rewritten. The binomial coefficient can be expressed using the gamma function:

\[
\binom{r+k-1}{k} = \frac{\Gamma(r+k)}{\Gamma(r) \Gamma(k+1)}
\]

Therefore, the PMF becomes:

\[
P(X = k) = \frac{\Gamma(r+k)}{\Gamma(r) \Gamma(k+1)} (1-p)^r p^k
\]

\section*{Simplification}

To simplify further, let us transform the variables and assume a new parameter \(n = k\):

\[
P(X = n) = \frac{\Gamma(r+n)}{\Gamma(r) \Gamma(n+1)} (1-p)^r p^n
\]

Using the property of the gamma function \(\Gamma(n+1) = n!\), we have:

\[
P(X = n) = \frac{\Gamma(r+n)}{\Gamma(r) n!} (1-p)^r p^n
\]

Now, we can see that:

\[
\Gamma(r+n) = (r+n-1) \cdot (r+n-2) \cdots (r) \cdot \Gamma(r)
\]

When $n \rightarrow \infty$, this can be approximated by \(n^{r-1} \cdot \Gamma(r)\), thus:

\[
P(X = n) \approx \frac{(1-p)^r p^n n^{r-1} \Gamma(r)}{\Gamma(r) n!}
\]

Now, We can see that:

\[
P(X = n) \approx \frac{(1-p)^r \cdot n^{r-1} \cdot p^n}{\Gamma(r)}
\]

This is the relationship between the negative binomial distribution and the gamma. Now we can use it in the book example.
	\end{solve}
	% =================================================
	
	% \newpage
	
	% \vfill
	
	\bibliographystyle{apalike}
	\bibliography{references}
\end{document}