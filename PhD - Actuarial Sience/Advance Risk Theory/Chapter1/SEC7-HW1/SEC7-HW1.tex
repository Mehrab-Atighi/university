\documentclass[12pt]{article}
\usepackage{amsmath,amsfonts,amssymb,graphicx,geometry,booktabs}
\geometry{a4paper, margin=1in}
\title{The Cram\'er-Lundberg Model in Risk Theory}
\author{Prepared for Academic Submission}
\date{\today}

\begin{document}
	
	\maketitle
	
	\begin{abstract}
		This document provides an overview of the Cram\'er-Lundberg model, a cornerstone in risk theory. It includes mathematical foundations, implementation details, and connections to simulations using the Shiny application. The focus is on analyzing risk processes under Lognormal and Pareto claim size distributions, emphasizing how parameter variations influence the variance and behavior of the risk process.
	\end{abstract}
	
	\section{Introduction}
	The Cram\'er-Lundberg model is a fundamental stochastic model in risk theory, used to describe the surplus of an insurance company over time. The surplus process is defined as:
	\[
	R(t) = u + ct - S(t), \quad t \geq 0,
	\]
	where:
	\begin{itemize}
		\item $u$ is the initial reserve,
		\item $c$ is the premium rate,
		\item $S(t)$ is the aggregate claim amount by time $t$.
	\end{itemize}
	The goal is often to study the probability of ruin, which occurs when $R(t) < 0$ for some $t \geq 0$.
	
	\section{Mathematical Formulation}
	\subsection{Aggregate Claims}
	The aggregate claims $S(t)$ are modeled as:
	\[
	S(t) = \sum_{i=1}^{N(t)} X_i,
	\]
	where:
	\begin{itemize}
		\item $N(t)$ is a Poisson process with rate $\lambda$, representing the number of claims up to time $t$,
		\item $X_i$ are independent and identically distributed random variables representing claim sizes.
	\end{itemize}
	The expected value and variance of $S(t)$ are:
	\[
	\mathbb{E}[S(t)] = \lambda t \mu, \quad \text{Var}(S(t)) = \lambda t \sigma_X^2,
	\]
	where $\mu = \mathbb{E}[X_i]$ and $\sigma_X^2 = \text{Var}(X_i)$.
	
	\subsection{Risk Process}
	The risk process is given by:
	\[
	R(t) = u + ct - \sum_{i=1}^{N(t)} X_i.
	\]
	If $\mathbb{E}[X_i] = \mu$, the expected value of the risk process is:
	\[
	\mathbb{E}[R(t)] = u + ct - \lambda \mu t.
	\]
	
	\subsection{Probability of Ruin}
	The probability of ruin $\psi(u)$ is defined as:
	\[
	\psi(u) = \mathbb{P}(\exists t \geq 0 \text{ such that } R(t) < 0).
	\]
	Under the Cram\'er-Lundberg approximation:
	\[
	\psi(u) \approx Ce^{-\gamma u},
	\]
	where $\gamma$ is the adjustment coefficient satisfying:
	\[
	\mathbb{E}[e^{-\gamma X}] = 1 - \frac{c}{\lambda}.
	\]
	
	\section{Claim Size Distributions}
	\subsection{Lognormal Distribution}
	The Lognormal distribution is defined as:
	\[
	X \sim \text{Lognormal}(\mu, \sigma),
	\]
	with probability density function:
	\[
	f_X(x) = \frac{1}{x \sigma \sqrt{2\pi}} \exp\left(-\frac{(\ln x - \mu)^2}{2\sigma^2}\right), \quad x > 0.
	\]
	The mean and variance are given by:
	\[
	\mathbb{E}[X] = e^{\mu + \frac{\sigma^2}{2}}, \quad \text{Var}(X) = \left(e^{\sigma^2} - 1\right)e^{2\mu + \sigma^2}.
	\]
	As $\sigma$ increases, the variance of $X$ grows exponentially, leading to greater variability in the risk process.
	
	\subsection{Pareto Distribution}
	The Pareto distribution is defined as:
	\[
	X \sim \text{Pareto}(\alpha, x_m),
	\]
	with probability density function:
	\[
	f_X(x) = \frac{\alpha x_m^\alpha}{x^{\alpha + 1}}, \quad x \geq x_m > 0.
	\]
	The mean exists for $\alpha > 1$ and is:
	\[
	\mathbb{E}[X] = \frac{\alpha x_m}{\alpha - 1},
	\]
	and the variance exists for $\alpha > 2$:
	\[
	\text{Var}(X) = \frac{\alpha x_m^2}{(\alpha - 1)^2(\alpha - 2)}.
	\]
	As $\alpha$ decreases, the variance increases significantly, leading to heavier tails and a higher probability of extreme claim sizes.
	
	\section{Implementation in Shiny}
	The Shiny application simulates the risk process under Lognormal and Pareto distributions, with parameters adjustable via the interface. Changes to parameters such as $\lambda$, $\mu$, $\sigma$, $\alpha$, and $x_m$ affect the behavior and variance of the risk process.
	
	\subsection{Simulation Algorithm}
	The simulation involves:
	\begin{enumerate}
		\item Generating claim arrival times using a Poisson process with rate $\lambda$.
		\item Sampling claim sizes from the specified distribution.
		\item Calculating $S(t)$ and $R(t)$ for each time step.
	\end{enumerate}
	For instance, increasing $\lambda$ increases the expected number of claims, directly impacting the variance of $S(t)$ \cite{Embrechts.etal1997}.
	
	\subsection{Key Code Snippets}
	The function to simulate the risk process is implemented as follows:
	\begin{verbatim}
		simulate_risk_process <- function(u, c, t_max, lambda, claim_size_dist) {
			t <- seq(0, t_max)
			num_claims <- rpois(1, lambda = lambda) + 1
			claim_times <- sort(runif(num_claims, 0, t_max))
			claim_sizes <- claim_size_dist(num_claims)
			S_t <- rep(0, length(t))
			for (i in 1:num_claims) {
				S_t[t >= claim_times[i]] <- S_t[t >= claim_times[i]] + claim_sizes[i]
			}
			RU <- u + c * t - S_t
			return(data.frame(time = t, RU = RU))
		}
	\end{verbatim}
	
	\section{Results and Visualization}
	The Shiny application provides:
	\begin{itemize}
		\item Time series plots of $R(t)$ for multiple simulations.
		\item Comparison of risk processes under Lognormal and Pareto distributions.
	\end{itemize}
	Sample plots include:
	\begin{itemize}
		\item Risk processes simulated for Lognormal distribution.
		\item Risk processes simulated for Pareto distribution.
		\item Combined comparison plot.
	\end{itemize}
	
	\section{Conclusion}
	The Cram\'er-Lundberg model provides a robust framework for analyzing risk processes. Adjusting parameters like $\lambda$, $\mu$, $\sigma$, $\alpha$, and $x_m$ directly impacts the variance and tail behavior of the claim size distributions. Using Shiny, it is possible to visualize these effects, facilitating deeper insights into insurance risk management.
	
	\bibliographystyle{apalike}
	\bibliography{references}
\end{document}
