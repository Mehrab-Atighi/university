	
\input{preamble}
\usepackage{titlesec}
\usepackage[many]{tcolorbox}
\usepackage{graphics}
\usepackage{float}
\usepackage{multirow}
\usepackage{float}
\usepackage{caption}
\usepackage{amssymb}
\usepackage{amsmath}
\usepackage{lineno}
\usepackage{algorithm}
\usepackage{algorithmic}
\usepackage{multirow}
\usepackage{graphics}
\usepackage{bm}
\usepackage{float}
\usepackage{listings}
\usepackage{color}
\usepackage{graphicx}



% Adjust spacing after the chapter title
% \titlespacing*{<command>}{<left>}{<before-sep>}{<after-sep>}
\titlespacing*{\chapter}{0cm}{-2.0cm}{0.50cm}
\titlespacing*{\section}{0cm}{0.50cm}{0.25cm}

% Indent 
\setlength{\parindent}{0pt}
\setlength{\parskip}{1ex}

% --- Theorems, lemma, corollary, postulate, definition ---
% \numberwithin{equation}{section}


\newtcbtheorem[]{problem}{Problem}%
    {enhanced,
    colback = black!5, %white,
    colbacktitle = black!5,
    coltitle = black,
    boxrule = 0pt,
    frame hidden,
    borderline west = {0.5mm}{0.0mm}{black},
    fonttitle = \bfseries\sffamily,
    breakable,
    before skip = 3ex,
    after skip = 3ex
}{problem}

\newtcbtheorem[]{solve}{Solve}%
{enhanced,
	colback = black!5, %white,
	colbacktitle = black,
	coltitle=white,
	boxrule = 0pt,
	frame hidden,
	borderline west = {0.5mm}{0.0mm}{black},
	fonttitle = \bfseries\sffamily,
	breakable,
	before skip = 3ex,
	after skip = 3ex
}{solve}

\tcbuselibrary{skins, breakable}
\input{commands}


\begin{document}
		\begin{Large}
		\textsf{\textbf{Check slowly varying - $f(x) = x^\alpha \ln(1+x)$}}\\
		Section 5 - Home Work 5
	\end{Large}
	
	\vspace{1ex}
	
	\textsf{\textbf{Student:}} \text{Mehrab Atighi}, \href{mailto:mehrab.atighi@gmail.com}{\texttt{mehrab.atighi@gmail.com}}\\
	\textsf{\textbf{Lecturer:}} \text{Mohammad Zokaei}, \href{mailto:Zokaei@sbu.ac.ir}{\texttt{Zokaei@sbu.ac.ir}}
	
	
	\vspace{2ex}
	
	\begin{problem}{}{problem-label}
		Show that if we set $f(x) = x^\alpha \ln(1+x)$ then check that is this function slowly varying or not.\cite{Embrechts.etal1997}:
	\end{problem}
	
	\begin{solve}{}{solve-label}
	To determine if the function \( f(x) = x^\alpha \ln(1+x) \) is a slowly varying function, we use the definition of a slowly varying function. A function \( L(x) \) is slowly varying at infinity if for all \( a > 0 \):
	$$ \lim_{x \to \infty} \frac{L(ax)}{L(x)} = 1 $$
	Let's apply this to \( f(x) = x^\alpha \ln(1+x) \):
	\begin{enumerate}
		\item \textbf{Substitute \( ax \) into the function:}
		$$f(ax) = (ax)^\alpha \ln(1+ax)$$
		\item \textbf{Form the ratio \( \frac{f(ax)}{f(x)} \):}
		
		$$\frac{(ax)^\alpha \ln(1+ax)}{x^\alpha \ln(1+x)} = a^\alpha \frac{\ln(1+ax)}{\ln(1+x)}$$
		\item \textbf{Take the limit as \( x \) approaches infinity:}
		$$\lim_{x \to \infty} a^\alpha \frac{\ln(1+ax)}{\ln(1+x)}$$
	\end{enumerate}
	
	To evaluate this limit, consider the behavior of the logarithmic function for large \( x \). For large \( x \), \( \ln(1+x) \approx \ln(x) \). Thus, we can approximate:
	
	\[
	\frac{\ln(1+ax)}{\ln(1+x)} \approx \frac{\ln(ax)}{\ln(x)} = \frac{\ln(a) + \ln(x)}{\ln(x)} = 1 + \frac{\ln(a)}{\ln(x)}
	\]
	
	As \( x \) approaches infinity, \( \frac{\ln(a)}{\ln(x)} \) approaches 0. Therefore:
	
	\[
	\lim_{x \to \infty} \frac{\ln(1+ax)}{\ln(1+x)} = 1
	\]
	
	Thus:
	
	\[
	\lim_{x \to \infty} a^\alpha \frac{\ln(1+ax)}{\ln(1+x)} = a^\alpha
	\]
	
	Since \( a^\alpha \neq 1 \) for \( \alpha \neq 0 \), \( f(x) = x^\alpha \ln(1+x) \) is **not** a slowly varying function unless \( \alpha = 0 \).
	
	I hope this helps! If you have any more questions or need further clarification, feel free to ask. \cite{r1,r2,r3,r4,r5,r6,r7,r8,r9,10,r11}
	\end{solve}
	% =================================================
	
	% \newpage
	
	% \vfill
	
	\bibliographystyle{apalike}
	\bibliography{references}
\end{document}