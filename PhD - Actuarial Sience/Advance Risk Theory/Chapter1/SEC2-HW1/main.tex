	
\input{preamble}
\usepackage{titlesec}
\usepackage[many]{tcolorbox}
\usepackage{graphics}
\usepackage{float}
\usepackage{multirow}
\usepackage{float}
\usepackage{caption}
\usepackage{amssymb}
\usepackage{amsmath}
\usepackage{lineno}
\usepackage{algorithm}
\usepackage{algorithmic}
\usepackage{multirow}
\usepackage{graphics}
\usepackage{bm}
\usepackage{float}
\usepackage{listings}
\usepackage{color}
\usepackage{graphicx}



% Adjust spacing after the chapter title
% \titlespacing*{<command>}{<left>}{<before-sep>}{<after-sep>}
\titlespacing*{\chapter}{0cm}{-2.0cm}{0.50cm}
\titlespacing*{\section}{0cm}{0.50cm}{0.25cm}

% Indent 
\setlength{\parindent}{0pt}
\setlength{\parskip}{1ex}

% --- Theorems, lemma, corollary, postulate, definition ---
% \numberwithin{equation}{section}


\newtcbtheorem[]{problem}{Problem}%
    {enhanced,
    colback = black!5, %white,
    colbacktitle = black!5,
    coltitle = black,
    boxrule = 0pt,
    frame hidden,
    borderline west = {0.5mm}{0.0mm}{black},
    fonttitle = \bfseries\sffamily,
    breakable,
    before skip = 3ex,
    after skip = 3ex
}{problem}

\newtcbtheorem[]{solve}{Solve}%
{enhanced,
	colback = black!5, %white,
	colbacktitle = black,
	coltitle=white,
	boxrule = 0pt,
	frame hidden,
	borderline west = {0.5mm}{0.0mm}{black},
	fonttitle = \bfseries\sffamily,
	breakable,
	before skip = 3ex,
	after skip = 3ex
}{solve}

\tcbuselibrary{skins, breakable}
\input{commands}


\begin{document}
		\begin{Large}
		\textsf{\textbf{Homogeneous Poisson process}}\\
		Section1 - HW1
	\end{Large}
	
	\vspace{1ex}
	
	\textsf{\textbf{Student:}} \text{Mehrab Atighi}, \href{mailto:mehrab.atighi@gmail.com}{\texttt{mehrab.atighi@gmail.com}}\\
	\textsf{\textbf{Lecturer:}} \text{Mohammad Zokaei}, \href{mailto:Zokaei@sbu.ac.ir}{\texttt{Zokaei@sbu.ac.ir}}
	
	
	\vspace{2ex}
	
	\begin{problem}{}{problem-label}
		A consequence of the Cramer-Lundberg model definition is that ($N(t)$) is a homogeneous Poisson process with intensity $\lambda > 0$. Hence \cite{Embrechts.etal1997}:
		
		\begin{enumerate}[(a)]
			\item Prove the following
			\begin{enumerate}[label = (\roman*)]
				\item $P(N(t) =k) = \exp^{-\lambda t} \frac{(\lambda t)^k}{k!} \, k = 0,1,2,\cdots$ .
			\end{enumerate}
		\end{enumerate}
	\end{problem}
	
	\begin{solve}{}{solve-label}
		we know that Any counting process N(t) must satisfy:\\
		\begin{enumerate}
			\item N(t) $\geq$ 0;
			\item N(t) is integer valued;
			\item if s<t, then N(s) $\leq$ N(t);
			\item For any s<t, N(t) $\leq$ N(s) equals the number of events that occur in the interval $(s,t]$
		\end{enumerate}
		Consider a Poisson process:\\
		\begin{enumerate}
			\item 	Denote the time of the first event by $T_1$.
			\item 	For any n>1, let $T_n$ denote the elapsed time between the $(n-1)$st and the $n$th event.
		\end{enumerate}
		The sequence ${T_n,n = 1,2,\cdots} $is called the sequence of \textbf{interarrival times}.

				%\captionsetup{justification=centering}
				\includegraphics[width=\textwidth]{pic1.png}
				%\caption{ \textbf{interarrival times}}
				\label{fig1}
		
		and we know that ${T_n,n=1,2,\cdots}$ are independent identically distributed (iid) exponential random variables with parameter $\lambda$.\\
		$$P(T_1 > t) = P(N(t) = 0) = \exp^{\lambda t} \rightarrow T_1 \sim \text{EXP}(\lambda )$$
		
		The total waiting time for n occurrences of the event has a Gamma distribution (with parameters $(n,\lambda)$
		\\
		This implies that
		$$E(S_n) = \frac{n}{\lambda} \, Var(S_n) = \frac{n}{\lambda^2}$$
		
		Let 
		$$N(t) = \max\{n\geq0:T_1 + \cdots + T_n \leq t\}$$
		Then {$N(t), t\geq $} is a Poisson process with rate $\lambda$.
		for show the above relation we have:\\
		Fix an integer $n\geq 0$. Then $S_n = T_1 + \cdots + T_n \sim \Gamma(n,\lambda) $ and it is independent of $T_{n+1}.$\\
		By definition of \begin{align*}
			P(N(t) = n) &= P(S_n \leq t, S_n + T_{n+1} > t) \\
			&= \int_0^t \int_{t-s}^{\infty} f_{S_n}(s) f_{T_{n+1}}(x) \, dx \, ds \\
			&= \int_0^t P(T_{n+1} > t - s) f_{S_n}(s) \, ds \\
			&= \int_0^t e^{-\lambda(t - s)} \frac{\lambda(\lambda s)^{n-1} e^{-\lambda s}}{(n-1)!} \, ds \\
			&= \frac{(\lambda t)^n e^{-\lambda t}}{n!}.
		\end{align*}
		This shows that $N(t) \sim Pois(\lambda t)$
		The homogeneous Poisson process is a type of stochastic process that models events occurring randomly over time. Let's go through a step-by-step proof of some fundamental properties of a \textbf{homogeneous Poisson process} $N(t)$, with rate $\lambda > 0$. \cite{LectureA}
		
		Another way to solve this question is that:
		\begin{enumerate}
			\item Definition\\
			A \textbf{Poisson process} $N(t)$ with rate $\lambda > 0$ is defined as a stochastic process with the following properties:
			\subitem 1. $N(0) = 0$ (the process starts at 0).
			\subitem 2. \textbf{Independent increments}: The number of events that occur in disjoint time intervals are independent.
			\subitem 3. \textbf{Stationary increments}: The probability of $k$ events occurring in any time interval of length \( t \) depends only on $t$ , not on where the interval starts, and is given by the Poisson distribution:
			$$P(N(t + s) - N(s) = k) = \frac{(\lambda t)^k e^{-\lambda t}}{k!}, \quad k = 0, 1, 2, \dots$$
			
			\item Proof: $N(t) \sim \text{Poisson}(\lambda t)$
			
			We will prove that the number of events $N(t)$ in a time interval $[0, t]$ follows a Poisson distribution with parameter $\lambda t$.
			
			\subitem Step 1: Small time intervals approximation
			
			Divide the time interval $[0, t]$ into $n$ small sub-intervals of length  $\Delta t = \frac{t}{n}$. For large $n$, each sub-interval is short, and we assume that:
			\subsubitem 1. The probability of one event occurring in a sub-interval is approximately $\lambda \Delta t$.
			\subsubitem 2. The probability of more than one event occurring in a sub-interval is negligible, i.e., $O(\Delta t^2)$.
			
			Thus, for each sub-interval $[t_i, t_{i+1}]$:
			$$P(\text{1 event in } [t_i, t_{i+1}]) \approx \lambda \Delta t, \quad P(\text{no event in } [t_i, t_{i+1}]) \approx 1 - \lambda \Delta t.$$
			
			\subitem Step 2: Approximation for the total number of events
			
			Let $N_n(t)$ represent the number of events in the $n$ sub-intervals. Since the intervals are independent, $N_n(t)$ is the sum of $n$ independent Bernoulli random variables, where the probability of an event in each sub-interval is $\lambda \Delta t$.
			
			The expected number of events in $[0, t]$ is:
			
			$$E[N_n(t)] = n \cdot \lambda \Delta t = \lambda t$$.
			
			
			As $n \to \infty$, the sum of these Bernoulli trials converges to a Poisson distribution with mean $\lambda t$ (this follows from the \textbf{Poisson limit theorem}).
			
			\item Step 3: Deriving the Poisson distribution
			From the Poisson limit theorem, we conclude that as $\Delta t \to 0$  (or equivalently $n \to \infty$ ), the number of events $N(t)$ in $[0, t]$ converges to a Poisson random variable with parameter $\lambda t$, i.e.,
			
			$$P(N(t) = k) = \frac{(\lambda t)^k e^{-\lambda t}}{k!}, \quad k = 0, 1, 2, \dots$$
			
		\end{enumerate}
		\textbf{Conclusion:}\\
		We have shown that the number of events in a homogeneous Poisson process over a time interval $[0, t]$ follows a Poisson distribution with mean $\lambda t$, confirming the definition of the homogeneous Poisson process.
	\end{solve}
	% =================================================
	
	% \newpage
	
	% \vfill
	
	\bibliographystyle{apalike}
	\bibliography{references}
\end{document}