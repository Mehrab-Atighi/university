\documentclass[letterpaper, 11pt]{extarticle}
% \usepackage{fontspec}

% ==================================================

% document parameters
% \usepackage[spanish, mexico, es-lcroman]{babel}
\usepackage[english]{babel}
\usepackage[margin = 1in]{geometry}

% ==================================================

% Packages for math
\usepackage{mathrsfs}
\usepackage{amsfonts}
\usepackage{amsmath}
\usepackage{amsthm}
\usepackage{amssymb}
\usepackage{physics}
\usepackage{dsfont}
\usepackage{esint}

% ==================================================

% Packages for writing
\usepackage{enumerate}
\usepackage[shortlabels]{enumitem}
\usepackage{framed}
\usepackage{csquotes}

% ==================================================

% Miscellaneous packages
\usepackage{float}
\usepackage{tabularx}
\usepackage{xcolor}
\usepackage{multicol}
\usepackage{subcaption}
\usepackage{caption}
\captionsetup{format = hang, margin = 10pt, font = small, labelfont = bf}

% Citation
\usepackage[round, authoryear]{natbib}

% Hyperlinks setup
\usepackage{hyperref}
\definecolor{links}{rgb}{0.36,0.54,0.66}
\hypersetup{
   colorlinks = true,
    linkcolor = black,
     urlcolor = blue,
    citecolor = blue,
    filecolor = blue,
    pdfauthor = {Author},
     pdftitle = {Title},
   pdfsubject = {subject},
  pdfkeywords = {one, two},
  pdfproducer = {LaTeX},
   pdfcreator = {pdfLaTeX},
   }
\usepackage{titlesec}
\usepackage[many]{tcolorbox}

% Adjust spacing after the chapter title
% \titlespacing*{<command>}{<left>}{<before-sep>}{<after-sep>}
\titlespacing*{\chapter}{0cm}{-2.0cm}{0.50cm}
\titlespacing*{\section}{0cm}{0.50cm}{0.25cm}

% Indent 
\setlength{\parindent}{0pt}
\setlength{\parskip}{1ex}

% --- Theorems, lemma, corollary, postulate, definition ---
% \numberwithin{equation}{section}


\newtcbtheorem[]{problem}{Problem}%
    {enhanced,
    colback = black!5, %white,
    colbacktitle = black!5,
    coltitle = black,
    boxrule = 0pt,
    frame hidden,
    borderline west = {0.5mm}{0.0mm}{black},
    fonttitle = \bfseries\sffamily,
    breakable,
    before skip = 3ex,
    after skip = 3ex
}{problem}

\tcbuselibrary{skins, breakable}
%%% Commands defined by me
%   Euler constant
\newcommand{\eu}{\mathrm{e}}

%   Imaginary unit
\newcommand{\im}{\mathrm{i}}

%   Degrees
\newcommand{\grado}{\,^{\circ}}

%%% Linear Algebra

% Transpose
\newcommand{\transpose}[1]{{#1}^{\mathsf{T}}}

%%% Calculus
%   Integral from - infinity to infinity 
\newcommand{\Int}{\int\limits_{-\infty}^{\infty}}

%   Indefinite integral
\newcommand{\rint}[2]{\int{#1}\dd{#2}}

%   Definite integral
\newcommand{\Rint}[4]{\int\limits_{#1}^{#2}{#3}\dd{#4}}


% Serif bold text
\newcommand{\tsb}[1]{\textsf{\textbf{#1}}}

% Separation line
\newcommand{\linea}{\textcolor{gray!60}{\rule{\linewidth}{0.2pt}}}

% Bigger version of a \cdot to denote dot product
\makeatletter
\newcommand*\bigcdot{\mathpalette\bigcdot@{.5}}
\newcommand*\bigcdot@[2]{\mathbin{\vcenter{\hbox{\scalebox{#2}{$\m@th#1\bullet$}}}}}
\makeatother

% My Hamiltonian prefered notation
\newcommand{\Ham}{\hat{\mathcal{H}}}

%% Pre-existing commands redefined by me
% Trace of a matrix 
\renewcommand{\Tr}{\mathrm{Tr}}

\usepackage{actuarialsymbol}

\begin{document}
	\begin{Large}
		\textsf{\textbf{Type of Premium and policy value calculations}}\\
		Example 7.18
	\end{Large}
	
	\vspace{1ex}
	
	\textsf{\textbf{Student:}} \text{Mehrab Atighi}, \href{mailto:mehrab.atighi@gmail.com}{\texttt{mehrab.atighi@gmail.com}}\\
	\textsf{\textbf{Teacher:}} \text{Shirin Shoaee}, \href{mailto:sh_shoaee@sbu.ac.ir}{\texttt{sh\_shoaee@sbu.ac.ir}}
	
	\vspace{2ex}
	
	\begin{problem}{}{Example 7.10}
		An insurer issues a whole life insurance policy to a life aged 50. The sum insured of 100 000\$ is payable at the end of the year of death. Level premiums are payable annually in advance throughout the term of the contract. All premiums and policy values are calculated using the Standard Select Survival Model, and an interest rate of 5\% per year effective. Initial expenses are 50\% of the gross premium plus 250\$. Renewal expenses are 3\% of the gross premium plus 25\$ at each premium date after the first.\\
		Calculate:\\
		(a) the expense loading, $P^e$, and\\
		(b) $_{10}V^e$ , $_{10}V^n$ and $_{10}V^g$.
	\end{problem}
	
	\begin{solve}{}{Example 7.18 solution}
		\section*{Solution}
		
		\subsection*{Step 1: Define the Random Loss Variable}
		The random loss variable $L_t$ at time $t$ is defined as:
		\[
		L_t = B_t - P_t - E_t,
		\]
		where:
		\begin{itemize}
			\item $B_t$: the benefits variable at time $t$,
			\item $P_t$: the premiums variable  at time $t$,
			\item $E_t$: the expenses variable at time $t$.
		\end{itemize}
		
		\subsection*{Step 2: Components of the Loss Variables}
		\subsubsection*{1. Benefits}
		The benefits are \$100,000, payable at the end of the year of death. The present value of benefits as time t is given by:
		\[
		B_t = 100,000 \cdot \sum_{k=1}^{n} v^{k} \cdot \,q_{x},
		\]
		where:
		\begin{itemize}
			\item $n = \infty $ (until death),
			\item $v = (1+i)^{-1}$ is the annullay discount factor, $i = 0.05$ annually,
			\item $q_{x}$ is the probability of dying in the 1 year.
		\end{itemize}
		
		\subsubsection*{2. Premiums}
		The annually premium is \$. The present value of premiums at time t is:
		\[
		P_t = \sum_{k=1}^{n}   \cdot v^{k} \cdot \,p_{x}.
		\]
		where:
		\begin{itemize}
			\item $n = \infty$ ,
			\item $p_{x}$ is the probability of survival a person with x age for 1 year.
		\end{itemize}
		
		\subsubsection*{3. Expenses}
			\begin{itemize}
			\item Inital expenses:\\
			according to the example information we have:
			$$0.5P^g + 250\$$$ 
			the above value will be pay just at t = 0.
		
			\item renewal expenses:\\
			we know that renewal expenses are :
			$0.03 p^g + 25\$$.
			\end{itemize}

		Expenses are 10\% of each gross premium paid, so the monthly expense is:
		\[
		\text{Expense per premium payment} = 0.1 \cdot P = 46.
		\]
		So if we want tot calculate the expense loading or $p^e$ we should calculate the gross premium at the step 1.
		
		
		\subsection*{Step 3: Expected Present Value (EPV) of the Loss Random Variable}
		Using the survival model, the gross premium policy value at time $t$ is the expected value of the loss random variable:
		\[
		V_t = \text{EPV}(Z_t) - \text{EPV}(P_t) - \text{EPV}(E_t).
		\]
		\subsection*{Step 4: Components of the Expected Present Value (EPV) of the Loss Random Variable}
		\subsubsection*{1. Expected Present Value of Benefits (\(Z_t\))}
		
		The benefits are 100,000\$, payable at the end of the month of death. The expected present value (EPV) of these benefits at time \(t\) is denoted by:
		\[
		EPV(B)_t = S \cdot \,  A_{[50]} = 100,000 \cdot \,  A_{[50]}.
		\]
		where:
		\begin{itemize}
			\item \(A_{[50]}\): The expected present value of a 1-unit insurance benefit, payable at the end of the year of death, for a life aged \(50\).
			\item The symbol \( A_{[50]}\) incorporates the annually interest rate and survival probabilities, as follows:
			\[
			A_{[50]} = \sum_{k=1}^{n} v^{k} \cdot \,q_{50}.
			\]
			\item \(v = (1+i)^{-1}\): The annual discount factor %, with \(i = 0.05\).
			\item \(n = \infty\): The total number of years for the whole life insurance.
			\item \(q_{50}\): The probability that the life dies in the \((k)\)th year for a person with age 50.
		\end{itemize}
		
		\subsubsection*{2. Expected Present Value of Premiums (\(P_t\))}
		
		The premiums are payable annually until death. The expected present value (EPV) of these premiums at time \(t\) is denoted by:
		\[
		EPV(P)_t = P \cdot \, \ddot{a}_{[50]} = P \cdot \, \ddot{a}_{[50]}.
		\]
		where:
		\begin{itemize}
			\item \(\ddot{a}_{[50]}\): The expected present value of a 1-unit premium paid annually in advance for a life aged \(50\), until death, with payments made 1 times per year(immediately).
			\item The symbol \(\ddot{a}_{[50]}\) incorporates the annually interest rate and survival probabilities, as follows:
			\[
			\sum_{k=1}^{\infty} v^{k} \cdot \,_kp_{50}
			\]
			where:
			\begin{itemize}
				\item \(v = (1+i)^{-1}\): The annual discount factor %, with \(i = 0.05\),
				\item \(n = \infty \): The total number of year premiums,
				\item \(_kp_{50}\): The probability that the life age 50 dont death until 50 + k age.
			\end{itemize}
		\end{itemize}
		
		\subsubsection*{3. Expected Present Value of Expenses (\(E_t\))}
		
		we know that at the t =0 there is no renewal expense because its the first time of contract and only we have initial expense but in the other premium date its equal to the $0.03 p^g + 25\$$ as an annually annuity until death. so we have two part for renewal calculation here, the first part is from t=1 to $\infty$ and the second part is t = 0 premium that we should not pay that. attention please:
		\[
		EPV(E)_t = (0.03 p^g + 25) \ddot{a}_{[50]} - (0.03 p^g + 25) + (0.5 p^g + 250) .
		\]
		where:
		\begin{itemize}
				\item \(\ddot{a}_{[50]}\): The expected present value of a 1-unit premium paid annually in advance for a life aged \(50\), until death, with payments made 1 times per year(immediately).
			\item The symbol \(p^g\) is gross premium.
		\end{itemize}
		
		
		\subsection*{Step 5: Premiums Value Calculation at Times \(t = 0\)}
		
		The policy value at any time \(t\) is given by the general formula:
		\[
		tV_x + EPV_t(\text{future premiums}) = EPV_t(\text{future benefits}) + EPV_t(\text{future expenses}),
		\]
		where:
		\begin{itemize}
			\item \(tV_x\): The policy value at time \(t\) for a life aged \(x+t\).
			\item \(EPV_t(\text{future premiums})\): The expected present value of premiums still to be paid after time \(t\).
			\item \(EPV_t(\text{future benefits})\): The expected present value of benefits payable after time \(t\).
			\item \(EPV_t(\text{future expenses})\): The expected present value of expenses after time \(t\) we should attention that the expenses are two part here, the first part is initial expenses and the second part is renewal expenses.
		\end{itemize}
		
		To calculate gross premium at $t= 0$ we should use general formula:
		\begin{small}
		\begin{align*}
		p^g \ddot{a_{[50]}} &= 100,000 \cdot \,  A_{[50]} + (0.03 p^g + 25) \ddot{a}_{[50]} - (0.03 p^g + 25) + (0.5 p^g + 250)\\
		<=> p^g \ddot{a_{[50]}} &= 100,000 \cdot \,  A_{[50]} + 0.03 p^g \ddot{a}_{[50]} + 25\ddot{a}_{[50]} - 0.03 p^g + 25 + 0.5 p^g + 250\\
		<=> p^g \ddot{a_{[50]}} &= 100,000 \cdot \,  A_{[50]} + 0.03 p^g \ddot{a}_{[50]} + 25\ddot{a}_{[50]} + 0.47 p^g + 225\\
		<=> p^g \ddot{a_{[50]}} - 0.03 p^g \ddot{a}_{[50]} - 0.47 p^g &= 100,000 \cdot \,  A_{[50]} + 25\ddot{a}_{[50]}  + 225\\
		<=> p^g(\ddot{a_{[50]}} - 0.03\ddot{a}_{[50]} - 0.47) &= 100,000 \cdot \,  A_{[50]} + 25\ddot{a}_{[50]}  + 225\\
		<=> p^g(0.97 \ddot{a_{[50]}} - 0.47) &= 100,000 \cdot \,  A_{[50]} + 25\ddot{a}_{[50]}  + 225\\
		<=> p^g &= \frac{100,000 \cdot \,  A_{[50]} + 25\ddot{a}_{[50]}  + 225}{(0.97 \ddot{a_{[50]}} - 0.47)}  = 1219.09 \$
	\end{align*}
		\end{small}
		Now \( P^e \) can be calculated by finding the EPV of future expenses, and calculating the level premium to fund those expenses – that is \[ P^e \ddot{a}_{[50]} = 25 \ddot{a}_{[50]} + 225 + 0.03 P^e \ddot{a}_{[50]} + 0.47 P^e. \] Alternatively, we can calculate the net premium, \[ P^n = 100000 A_{[50]}/\ddot{a}_{[50]} = \$1110.65, \] and use \( P^e = P^e - P^n \). Either method gives \( P^e = \$108.43 \). Compare the expense premium with the incurred expenses. The annual renewal expenses, payable at each premium date after the first, are \$61.57. The rest of the expense loading, \$46.86 at each premium date, reimburses the acquisition expenses, which total \$859.54.
		
	\subsection*{Step 6: Different Type of Policy Values Calculation at Times \(t = 10\)}
	we know that for calculating policy values we have 3 type of them according to the premiums.
	\begin{itemize}
		\item $_tV^n$ net premium policy value
		\item $_tV^g$ gross premium policy value
		\item $_tV^e$ expense policy value
		\end{itemize}
		and we have bottem formula according to the book:
		\begin{itemize}
		\item $_tV^n = EPV(future Benefits) - EPV(future net premium)$ 
		\item $_tV^g = EPV(future Benefits) + EPV(future expenses) - EPV(future gross premium)$ 
		\item $_tV^e = EPV(future expenses) - EPV(future expenses loading)$ 
		\item $ _tV^g = _tV^n + _tV^e$
		\end{itemize}
		so we can replace the above formula with calculated values at step 5. Thus,
			\begin{itemize}
			\item $_{10}V^e = 0.03 p^g \ddot{a}_{[60]} + 25\ddot{a}_{[60]} - p^e \ddot{a}_{[60]} = -46.86 \ddot{a}_{[60]} = -698.42 \$$ 
			\item $_{10}V^n = 100,000 \cdot \,  A_{[60]} - p^n \ddot{a}_{[60]} = 12474.94 \$$ 
			\item $_{10}V^g = 100,000 \cdot \,  A_{[60]} + 25 \ddot{a}_{[60]} - 0.97 p^g \ddot{a}_{[60]} = 11776.52 \$$ 
		\end{itemize}		
	\end{solve}
	
	\vspace{2ex}
	
	% =================================================
	
	% \newpage
	
	% \vfill
%	
%	\bibliographystyle{apalike}
%	\bibliography{references}
\end{document}
